%%%%%%%%%%%%%%%%%%%%%%%%%%%%%%%%%%%%%%%%%%%%%%%%%%
%% Bachelor's & Master's Thesis Template        %%
%% Copyleft by Dawid Weiss & Marta Szachniuk    %%
%% Faculty of Computing and Telecommunication   %%
%% Poznan University of Technology, 2020        %%
%%%%%%%%%%%%%%%%%%%%%%%%%%%%%%%%%%%%%%%%%%%%%%%%%%

\documentclass[english,bachelor,a4paper,oneside]{ppfcmthesis}


\usepackage[utf8]{inputenc}
\usepackage[OT4]{fontenc}

%--------------------------------------
% Strona tytułowa
%--------------------------------------

% Autorzy pracy, jeśli jest ich więcej niż jeden
% wstaw między nimi separator \and
\author{%
   Jakub Grabowski \album{151825} \and 
   Filip Kozłowski \album{151823} \and 
   Krzysztof Matyla \album{151778} \and 
   Igor Warszawski \album{151585}}
\authortitle{}                                % Do not change.

\title{Biometric identification of a smartphone user using graph neural networks}

% Your supervisor comes here.
\ppsupervisor{~dr hab.~inż.~Szymon Szczęsny, ~prof. PP} 

% Year of final submission (not graduation!)
\ppyear{2025}                                 


\begin{document}

% Front matter starts here
\frontmatter\pagestyle{empty}%
\maketitle\cleardoublepage%

%--------------------------------------
% Miejsce na kartę pracy dyplomowej
%--------------------------------------

\thispagestyle{empty}\vspace*{\fill}%
\begin{center}Tutaj będzie karta pracy dyplomowej;\\oryginał wstawiamy do wersji dla archiwum PP, w pozostałych kopiach wstawiamy ksero.\end{center}%
\vfill\cleardoublepage%

%--------------------------------------
% Spis treści
%--------------------------------------

\pagenumbering{Roman}\pagestyle{ppfcmthesis}%
\tableofcontents* 
\cleardoublepage % Zaczynamy od nieparzystej strony

%--------------------------------------
% Rozdziały
%--------------------------------------

%Najwygodniej jeśli każdy rozdział znajduje się w oddzielnym pliku
\mainmatter%

\chapter{Introduction}

Biometric data is a widely used -- especially on mobile devices -- for user authentication. It is also used for person recognition. As of 2020, the majority of smartphones had biometric sensors, such as fingerprint readers \cite{statista_biometric_phones_2025}. Many computers can also provide biometric authentication via face recognition, if connected to a webcam, e.g. via Windows Hello on Windows 10 or 11 \cite{microsoft_windows_hello_2025}. These are, however, not the only possible recognition or authentication methods that use biometric data.

The project aimed to develop a model, along with a corresponding mobile app, capable of recognizing users based on their biometric data, primarily derived from keystroke patterns. Participants in the study, conducted as a part of the project, provided their data by entering long stretches of text as testing data. Models were created for each user, with the standard model testing procedures and validations. A subgroup of the study participants was also asked to verify the model in real-life testing by writing short paragraphs in the application, which were sent to the server for user verification.

The scope of the work was to create a mobile application capable of gathering the keystroke data, which could then be used by the server to create Graph Neural Network (GNN) models tasked with recognizing the user as opposed to other possible users. Also in the scope was performing a study on a group of participants who provided the data for the project and participated in the application and model demonstration and testing.

The sources referenced in this thesis primarily fall into two categories: studies on keystroke data models and their effectiveness, and specialist literature concerning Graph Neural Networks.

The thesis has the following structure:
\begin{itemize}
    \item Chapter 2 consists of some theory concerning biometrics, especially in the context of user input data, with a small literature review about using biometrics for user recognition.
    \item Chapter 3 contains basic theoretics about Graph Convolutional Networks, which are used for user recognition in the model created for the project.
    \item Chapter 4 is an overview of the project, explaining its components and the relationships between them. It contains subchapters about project use cases, server architecture and mobile application architecture.
    \item Chapter 5 is concerned with the Neural Network model, its design and feature engineering.
    \item Chapter 6 contains results of the study conducted on the users, with subchapter dedicated to discussing the findings.
    \item Chapter 7 is a brief conclusion to the thesis.
\end{itemize}

The division of labor for this project was as follows:
\begin{itemize}
    \item Jakub Grabowski created the mobile application, set up and coordinated the project, and researched biometrics for the thesis paper. He wrote chapters 1, 2, 7 and parts of chapters 3 and 6.
    \item Filip Kozłowski created the server and integrated the GNN model with it. He also planned and implemented communication between the server and the application. He wrote parts of chapters 3, 4 and 6. 
    \item Krzysztof Matyla helped in creating the mobile application interface, provided testing for various parts of the project, and coordinated user testing. He wrote most of chapter 4.
    \item Igor Warszawski planned and implemented the GNN model used on the server. He also tested and validated the results, together with Filip Kozłowski. He wrote chapter 5 and parts of chapters 3 and 6.
\end{itemize}


\chapter{Biometrics in mobile devices - theory}

Fundamental to the goal of the project was the use of biometric data in user identification. Biometric data can be defined as measurements of some unique characteristics of an individual. These can largely be divided into two main categories: physiological data, which is the measurement of the inherent characteristics of an individual's body, such as a fingerprint, an iris scan or a face scan, and behavioral data, which measures the person's movements, behaviors, speech patterns etc. \cite{Abde2023}

One possible way to extract data from a person's behavior is via keystroke dynamics. This type of behavioral biometrics is aquired from a user by the means of a keyboard or other typing device and records and extracts features from the way the keyboard is used. Most commonly used and almost universally aplicable to any keyboard device is the measuremnt of timings between each character typed. If the user uses a physical keyboard, it is also convenient to derive the following features \cite{Shar2023}:
1. Hold Time -- time between key press and release
2. Down-Down Time -- time between first key press and second key press
3. Up-Up Time -- time between first key release and second key release
4. Up-Down Time -- time between first key release and second key press
5. Down-Up Time -- time between first key press and second key release.

With some keyboards it may be more difficult to gather all the possible features. Even basic feature, such as the hold time can prove difficult when using for example GBoard on mobile devices, which does not naturally send key press and key release information to the application. This information can thus only be gathered in approximation or by building another virtual keyboard application. This, however, has its drawbacks. The users are generally used to one type of keyboard (on mobile it may be for example GBoard or SwiftKey), so forcing them to use another type of keyboard may be detrimental.

While the model may be less accurate because of the lack of features, there can be some ways to mitigate it. Some other features can be added, which are largely specific to mobile devices, such as accelerometer data, or a larger sample can be used. A few of those options were considered by the researchers, and the results will be discussed in the next chapters (chapters 3.3 to 3.5).

The keystroke identification can also rely on other data gathered from the keyboard, such as the specifics of letters used, their average frequencies, most common connections between the letter or other statistics \cite{Wang2024}. These statistics can be modeled in many ways. If the average Up-Up Time between two keys is gathered from the data, a graph can be formed, having additional features as see fit by the designers. Such graphs were constructed for the Neural Network models constructed in this study, which will be discussed in the next chapter.

TODO for JG: continue the paragraphs.

\chapter{Graph Convolutional Networks - theory}

TODO for anyone/everyone (probably me and FK): how GNNs work, how GCNs work, how the networks can be constructed.

Graphs can defined as mathematical structures $$G$$ consisting of a set of vertices $$V$$, a set of edges $$E$$ and an incidence function $\phi$, along with many variations and generalizations to such structure, can be used for describing entities, which are related to each other in some way. An example of such model could be a computer network graph or citation network. Neurons can also be modelled in a similar way. Relation data can often be best described using such graphs. \cite{Lesk2024}

In modern Machine Learning, a popular type of Neural Network is a Convolutional Neural Network. Such networks generally operate on grids. A Convolutional Neural Network (CNN) has a fixed node ordering -- some input must firstly be mapped into a grid to be used with a CNN. There are ways to map many types of data into such format. For the scope of this project, \cite{Shar2023} uses such an approach to map keystroke data onto a grid, that is later used... (TODO: continue, read the paper, sth)

In this project, the goal was to use the graph networks that can naturally arise from keystroke data to -- on a graph level -- try to infere the users identity. (TODO: continue the paragraph)

TODO: how do GNNs work?

TODO: how do GCNs and GCN networks work? Network design is better left for project model section.

TODO: graph-level prediction for GCN network.

\chapter{Gathering keystroke data on mobile devices}

There are many ways to recognise a phone user using biometrics, such as scanning fingerprints or facial recognition. It is very useful for security purposes. The ease of use and reliability have made passwords less popular and led to their replacement by biometrics. However, since other biometric methods are also available, it is reasonable to test if biometrics derived from writing button press intervals and phone orientation could also be a reliable way to recognise the user. To collect data and test the results, the mobile application was created.
The main goal of the application is to gather data with an easy-to-use, intuitive interface, send the data to a server for training purposes, check if the model recognises the user. 

As previously stated, State of the Art models can actually perform well (FIXSOURCE) on such data. These models are however usually trained on data gathered from physical keyboards. Additionally, the Neural Network model created for user identification was chosen to be based on Graph Convolutional Networks, which differ from models used by many researchers in the past (FIXSOURCE).
Because of that, an important part of the project was a study of results and data gathered, which is presented in chapter 3.4 and 3.5.

TODO: find statistics and add them to sources

\section{Mobile application for data gathering and model testing}

The application was written for Android devices supporting Android 8.1 or newer. As of 2024*, more than 93\% of Android devices should be compatible. The Android platform was chosen, as it was easier to test on and find a study group of the Android users as opposed to the iOS users (according to * , significantly more people in Poland, where the researchers are based in, use Android devices).

Technology used in the mobile application itself was Jetpack Compose, which is quoted by Google to be "Android’s recommended modern toolkit for building native UI" (from site*). Language used was Kotlin. Persisence was achieved by using Android Room, which provided an abstraction layer over SQLite database, which was used for data collection.

TODO for KM: add statistics sources (Internet), some technicals about the inner workings of the app. How the data is stored, what is gathered and when. If you have any doubts, feel free to ask about any methods/composables. I will be updating method descriptions/docs soon -- JG

Model View Controller and DataStore...

Data was modeled as...

Data was saved...

Application design...

Training screen...

Testing screen...

Communications with the server...

Data sent to the server and downloaded locally...

Those should be subsections

\chapter{Graph Convolutional Network for user recognition}

TODO for IW: anything and everything about the model, the inner structure, the feature extraction can also go there, ask FK how you want to split these subjects up. FK will probably also check in with something here.

\chapter{Model fine-tuning and hyperparameters - metrics}

TODO for IW: write about the fine-tuning process, the metrics used and why are they used, cross-validations used etc. You can also post some hyperparam statistics here.

\chapter{Testing model on users}

TODO for xxx: when the tests are done (hopefully a week) we will discuss this. Also, there could be a part about many-users recognition.

\chapter{Conlusion}

TODO for JG: not needed now, will write after the user tests are done.


%--------------------------------------
% Literatura
%--------------------------------------

\bibliographystyle{plain}{\raggedright\sloppy\small\bibliography{bibliography}}

%--------------------------------------
% Dodatki
%--------------------------------------

\cleardoublepage\appendix%

%--------------------------------------
% Informacja o prawach autorskich
%--------------------------------------

\end{document}
