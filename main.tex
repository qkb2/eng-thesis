%%%%%%%%%%%%%%%%%%%%%%%%%%%%%%%%%%%%%%%%%%%%%%%%%%
%% Bachelor's & Master's Thesis Template        %%
%% Copyleft by Dawid Weiss & Marta Szachniuk    %%
%% Faculty of Computing and Telecommunication   %%
%% Poznan University of Technology, 2020        %%
%%%%%%%%%%%%%%%%%%%%%%%%%%%%%%%%%%%%%%%%%%%%%%%%%%

\documentclass[english,bachelor,a4paper,oneside]{ppfcmthesis}


\usepackage[utf8]{inputenc}
\usepackage[OT4]{fontenc}
\usepackage{graphicx}

%--------------------------------------
% Strona tytułowa
%--------------------------------------

% Autorzy pracy, jeśli jest ich więcej niż jeden
% wstaw między nimi separator \and
\author{%
   Jakub Grabowski \album{151825} \and 
   Filip Kozłowski \album{151823} \and 
   Krzysztof Matyla \album{151778} \and 
   Igor Warszawski \album{151585}}
\authortitle{}                                % Do not change.

\title{Biometric identification of a smartphone user using graph neural networks}

% Your supervisor comes here.
\ppsupervisor{~dr hab.~inż.~Szymon Szczęsny, ~prof. PP} 

% Year of final submission (not graduation!)
\ppyear{2025}                                 


\begin{document}

% Front matter starts here
\frontmatter\pagestyle{empty}%
\maketitle\cleardoublepage%

%--------------------------------------
% Miejsce na kartę pracy dyplomowej
%--------------------------------------

\thispagestyle{empty}\vspace*{\fill}%
\begin{center}Tutaj będzie karta pracy dyplomowej;\\oryginał wstawiamy do wersji dla archiwum PP, w pozostałych kopiach wstawiamy ksero.\end{center}%
\vfill\cleardoublepage%

%--------------------------------------
% Spis treści
%--------------------------------------

\pagenumbering{Roman}\pagestyle{ppfcmthesis}%
\tableofcontents* 
\cleardoublepage % Zaczynamy od nieparzystej strony

%--------------------------------------
% Rozdziały
%--------------------------------------

%Najwygodniej jeśli każdy rozdział znajduje się w oddzielnym pliku
\mainmatter%

\chapter{Introduction}

Biometric data is a widely used -- especially on mobile devices -- for user authentication. It is also used for person recognition. As of 2024, most smartphones have biometric sensors, such as fingerprint readers (FIX SOURCE). Many computers also provide biometric authentication via face recognition (FIX SOURCE). These are, however, not the only possible recognition or authentication methods that use biometric data.

The project aimed to develop a model, along with a corresponding mobile app, capable of recognizing the user by their biometric data contained mostly within the keystroke data. The users in the study, which was a part of the project, provided their data by entering long stretches of text as testing data. Models were created for each user, with the standard model testing procedures and validations. A subgroup of the study participants was also asked to verify the model in real-life testing by writing short paragraphs in the application, which were sent to the server for user verification.

The scope of the work was to create a mobile application capable of gathering the keystroke data, which could then be used by the server to create Graph Neural Network (GNN) models tasked with recognizing the user as opposed to other possible users. Also in the scope was performing a study on a group of participants who provided the data for the project and participated in the application and model demonstration and testing.

The sources used in this thesis mostly concerned the two following groups: studies of keystroke data models and their effectiveness and the specialist literature on the topic of Graph Neural Networks.

The thesis has the following structure:
Chapter 2 consists of some theory concerning biometrics, especially in the context of user input data, with a small literature review about using biometrics for user recognition.
Chapter 3 contains basic theoretics about Graph Convolutional Networks, which are used for user recognition in the model created for the project.
Chapter 4 is a brief overview of the project, explaining its components and the relationships between them. It includes the following sections:
Section 4.1 consists of the description of the server.
Section 4.2 describes the mobile application used for user data collection and model validation.
Section 4.3 contains a description of the Neural Network model used for user recognition, complete with the hyperparameters used in model training and validation.
Section 4.4 describes the feature selection used for a model.
Section 4.5 discusses the metrics used in the model testing on data gathered from users and the testing results.
Section 4.6 concerns the user testing with the help of study participants and the study results.
Chapter 5 is a conclusion to the thesis.

Work on this project was divided as follows:
Jakub Grabowski created the mobile application, set up and coordinated the project, and researched biometrics for his thesis paper.
Filip Kozłowski created the server and integrated the GNN model with it. He also planned and implemented communication between the server and the application.
Krzysztof Matyla helped in creating the mobile application, provided testing for various parts of the project, and coordinated user testing.
Igor Warszawski planned and implemented the GNN model used on the server. He also tested and validated the results, together with Filip Kozłowski.

\chapter{Biometrics in mobile devices - theory}

Fundamental to the goal of the project was the use of biometric data in user identification. Biometric data can be defined as measurements of some unique characteristics of an individual. These can largely be divided into two main categories: physiological data, which is the measurement of the inherent characteristics of an individual's body, such as a fingerprint, an iris scan or a face scan, and behavioral data, which measures the person's movements, behaviors, speech patterns etc. \cite{Abde2023}

Uniqueness of one's body is well known in biology. Features that may be used for person's identification are for example (FIX SOURCE: https://www.biometricsinstitute.org/what-is-biometrics/types-of-biometrics/):

\begin{enumerate}
    \item \textbf{DNA} -- found in cells of the living organisms, this acid carries genetic information.
    \item \textbf{Eye features} -- human iris, retina and scleral veins can be used in eye scans.
    \item \textbf{Face} -- full face scan is often used for user recognition, for example in mobile devices and laptops. (FIX SOURCE: https://developers.google.com/ml-kit/vision/face-detection, https://support.apple.com/en-us/102381)
    \item \textbf{Fingerprints and finger shape} -- fingerprints are widely used in forensics (FIX SOURCE: https://www.nist.gov/forensic-biometrics) and in digital scanners on mobile devices and laptops.
\end{enumerate}

Other, less popular ways of identifying a person are for example: ear shape, gait, hand shape, heartbeat, keystroke dynamics, signatures, vein scans and voice recognition.

One possible way to extract data from a person's behavior is via *keystroke dynamics*. This type of behavioral biometrics is acquired from a user by means of a keyboard or other typing device and records and extracts features from the way the keyboard is used. Most commonly used and almost universally applicable to any keyboard device is the measurement of timings between each character typed. If the user uses a physical keyboard, it is also convenient to derive the following features \cite{Shar2023}:

\begin{enumerate}
    \item \textbf{Hold Time} -- time between key press and release
    \item \textbf{Down-Down Time} -- time between first key press and second key press
    \item \textbf{Up-Up Time} -- time between first key release and second key release
    \item \textbf{Up-Down Time} -- time between first key release and second key press
    \item \textbf{Down-Up Time} -- time between first key press and second key release.
\end{enumerate}

It can therefore be said that keystroke dynamics focus mostly on identifying user's rhythmic patterns in their keystrokes. Such data can be used in conjunction with for example a password or a passphrase as a means of additional protection against password theft -- this idea was already being tested in 1990 by Joyce and Gupta (FIX SOURCE). The concept itself was already being studied in 1980, with Gaines et al. (FIX SOURCE) experimenting on secretaries' typing latency data. By 1997, clustering methods were already being used in experiments on user data on a small scale (42 profiles) by Monrose et al. \cite{Monr1997}. Algorithms used for such data evolved after that point and the raise in popularity of neural networks.

Lu et al. \cite{Lu2020} used a combined CNN+RNN approach to obtain the results of ERR of 2.36\% on Buffalo dataset and 5.97\% on Clarkson II datasets. Çeker and Upadhyaya (FIX SOURCE) used a CNN to create a multiple classification model -- this method is mostly suitable for smaller datasets with smaller number of users, as opposed to creating a personalized model for each user. This method had an ERR of 2.3\%. 

Another aspect of the keystroke dynamics recognition methods is how and when the data is collected. The systems can either work with some specific strings being typed by the user (like in Çeker and Upadhyaya case) or with the users being free to type anything within some length constraints (like in Lu 2020 \cite{Lu2020}). In this project, the second approach was chosen as the more realistic one.

With some keyboards it may be more difficult to gather all the possible features. Even basic feature, such as the hold time can prove difficult to gather when using for example GBoard on mobile devices, which does not naturally send key press and key release information to the application (SOURCE). This information can thus only be gathered in approximation or by building another virtual keyboard application. This, however, has its drawbacks. The users are generally used to one type of keyboard (on mobile it may be for example GBoard or SwiftKey), so forcing them to use another type of keyboard may be detrimental. Same person may write somewhat differently on different keyboards and machines. This study includes a small subsection on cross-smartphone compatibility of the model, for example concerning two users using each others' smartphones.

While the model may be less accurate because of the lack of features, there can be some ways to mitigate it. Some other features can be added, which are largely specific to mobile devices, such as accelerometer data, or a larger sample can be used. A few of those options were considered by the researchers, and the results will be discussed in the next chapters (???).

The keystroke identification can also rely on other data gathered from the keyboard, such as the specifics of letters used, their average frequencies, most common connections between the letter or other statistics \cite{Wang2024}. These statistics can be modeled in many ways. If the average Up-Up Time between two keys is gathered from the data, a graph can be formed, having additional features as see fit by the designers. Such graphs were constructed for the Neural Network models constructed in this study, which will be discussed in the next chapter.

According to EU guidelines, all data used in Machine Learning models should also be ethically sourced (FIX SOURCE: EU GUIDELINES \url{https://eur-lex.europa.eu/legal-content/EN/TXT/?uri=CELEX%3A32024R1689}). In this study, all data was sourced from willing participants and anonymized using unique ID numbers given to the participants by the researchers.

\chapter{Graph Convolutional Networks - theory}

Graphs can defined as mathematical structures $G$ consisting of a set of vertices $V$, a set of edges $E$ and an incidence function $\phi$, along with many variations and generalizations to such structure, can be used for describing entities, which are related to each other in some way. An example of such model could be a computer network graph or citation network. Neurons can also be modelled in a similar way. Relation data can often be best described using such graphs. \cite{Lesk2024}

Some problems relating to such data can be solved using Convolutional Neural Networks -- this can also be the case for keystroke dynamics data, such as with Lu et al. \cite{Lu2020} or Sharma et al. \cite{Shar2023}. However, it can be reasoned that the Graph Neural Networks can also perform such tasks, with connections in graph data being used more directly in the model itself.

\section{Graph Neural Networks}
TODO: how do GNN graph embeddings etc. work?

\section{Convolutional Networks and Graph Convolutional Networks}
TODO: how do GCNs and GCN networks differ? Network design is better left for project model section.

\section{Graph-level prediction in GNN}
TODO: graph-level prediction for GCN network.

\chapter{Gathering keystroke data on mobile devices}

There are many ways to recognise a phone user using biometrics, such as scanning fingerprints or facial recognition. It is very useful for security purposes. The ease of use and reliability have made passwords less popular and led to their replacement by biometrics. However, since other biometric methods are also available, it is reasonable to test if biometrics derived from writing button press intervals and phone orientation could also be a reliable way to recognise the user. To collect data and test the results, the mobile application was created.
The main goal of the application is to gather data with an easy-to-use, intuitive interface, send the data to a server for training purposes, check if the model recognises the user. 

As previously stated, State of the Art models can actually perform well (FIXSOURCE) on such data. These models are however usually trained on data gathered from physical keyboards. Additionally, the Neural Network model created for user identification was chosen to be based on Graph Convolutional Networks, which differ from models used by many researchers in the past (FIXSOURCE).
Because of that, an important part of the project was a study of results and data gathered, which is presented in chapter 3.4 and 3.5.

TODO: find statistics and add them to sources



\section{Use cases}
The main goal of this application was the identification of users based on their distinctive typing behavior, which is known as keystroke dynamics. By using machine learning models and encrypted server transmission for analyzing the collected data, the application aimed to provide an additional layer of security beyond passwords or basic biometrics. This project tried to establish whether this type of behavioral biometric can be a reliable way of user authentication. \newline
The following use cases illustrate how the user would interact with the application and its features.

\begin{itemize}
	\item \textbf{Logging into the application}
	\begin{itemize}
		\item \textbf{Purpose:} Allowing the user to log in and associate the application with a predefined \texttt{ID}.
		\item \textbf{Steps:}
		\begin{itemize}
			\item The user opens the application.
			\item On the \texttt{Login Screen}, the user enters their \texttt{ID} in the input field.
			\item The user clicks the \texttt{Login} button to proceed.
			\item The application stores given \texttt{ID} for further operations.
		\end{itemize}
		\item \textbf{Result:} The user is logged in and is redirected to the \texttt{Home Screen}.
	\end{itemize}
	
	\item \textbf{Data collection from key presses}
	\begin{itemize}
		\item \textbf{Purpose:} Storing users' keyboards interaction data for analysis.
		\item \textbf{Steps:}
		\begin{itemize}
			\item The user navigates to \texttt{Training Screen}.
			\item The user types a predefined number of characters in total throughout 5 phases to complete training.
			\item The application registers data for every key press, including:
			\begin{itemize}
				\item Key ID (e.g., \texttt{A}, \texttt{h}, \texttt{3})
				\item Timestamp of key press action
				\item Press duration
				\item Accelerator data (X, Y, Z axis)
			\end{itemize}
			\item Data is stored in \texttt{KeyPressEntity} object for further processing.
		\end{itemize}
		\item \textbf{Result:} Full key press data is saved in the application, ready to be transformed into \texttt{TSV} format, transmitted to the server, or stored locally.
	\end{itemize}
	
	\item \textbf{Testing how well the model recognizes the user}
	\begin{itemize}
		\item \textbf{Purpose:} Verifying if the user entering data is the one associated with their \texttt{ID}.
		\item \textbf{Steps:}
		\begin{itemize}
			\item The user navigates to \texttt{Testing Screen}.
			\item The user types a predefined number of characters into the input field.
			\item The application registers the key press data.
			\item The data is transformed into TSV format, stored locally, and sent to the server:
			\begin{itemize}
				\item The application ensures the connection is secured with \texttt{SSL/TLS}.
				\item The \texttt{POST} method is used to send the data.
			\end{itemize}
			\item The server processes the data using the trained model.
			\item The server sends back a response to the application, including:
			\begin{itemize}
				\item Information on whether the user was recognized.
				\item The percentage of compliance with the user.
			\end{itemize}
		\end{itemize}
		\item \textbf{Result:} The application displays the result to the user.
	\end{itemize}
	
\end{itemize}

\section{Server structure and communication with the application}

TODO for FK: technical docs for the server and connections between the app and the server. Data layer can also be touched a little.

Mobile application can communicate with the server, which can be locally hosted on a computer. The programmer needs to...

Server uses FastAPI, which is...

Server has the following endpoints, which are used by application for...

\section{Mobile application for data gathering and model testing}

The application was written for Android devices supporting Android 8.1 or newer. As of 2024 \cite{androidStats}, more than 93\% of Android devices should be compatible. The Android platform was chosen, as it was easier to test on and find a study group of the Android users as opposed to the iOS users (according to \cite{operatingSystemDistribution} , significantly more people in Poland, where the researchers are based in, use Android devices).

Technology used in the mobile application itself was Jetpack Compose, which is quoted by Google to be "Android’s recommended modern toolkit for building native UI" \cite{jetpackCompose}. Language used was Kotlin. Persisence was achieved by using Android Room, which provided an abstraction layer over SQLite database, which was used for data collection.

\subsection{Model View ViewModel and DataStore}

The application uses Model-View-ViewModel (MVVM) provided by Jetpack Compose design pattern to support a clear separation of concerns. 
\begin{itemize}
	\item \textbf{Model:} Data is modeled using \texttt{KeyPressEntity} class, which represents a single key press event. It includes: 
	\begin{itemize}
		\item \textbf{Key} (\texttt{String}): The key pressed by the user.
		\item \textbf{Press Time} (\texttt{Long}): The exact timestamp of the key press event.
		\item \textbf{Duration} (\texttt{Long}): The time elapsed since the last key press event.
		\item \textbf{Accelerometer Data} (\texttt{Float}): Not used currently but could be useful for the future development of the application.
	\end{itemize}
	\begin{figure}[H]
		\centering
		\includegraphics[width=0.8\linewidth]{images/DataModel.png}
		\caption{KeyPressEntity.kt}
		\label{fig:data_model_view}
	\end{figure}
	
	The \texttt{KeyPressEntity} is stored in a local SQLite database via Room.
	
	\item \textbf{View:} The user interface is implemented using \textbf{Jetpack Compose}, a declarative UI framework. Key components of the view contain:
	\begin{itemize}
		\item \textbf{Input Fields:} Lets users enter their credentials (University ID) and use the application for training or testing by pressing keys. 
		\item \textbf{Completion progress:} Informs users on what phase they are and displays progress of completion, linked to the \texttt{phasesCompleted} state in the \texttt{MainViewModel}.
		\item \textbf{Buttons:} Used for logging in, logging out, jumping phases and sending or downloading the data collected through training or testing stage.
	\end{itemize}
	\item \textbf{ViewModel:} This role is fulfilled by \texttt{MainViewModel}, which manages the application logic, handles interactions between the model and the view, and maintains the state of the app. \newline
	The \texttt{MainViewModel} class manages this operations through:
	\begin{itemize}
		\item \textbf{Logic Handling:} Methods such as \texttt{login()}, \texttt{logout()}, \texttt{clearDatabase()}, and \texttt{onKeyPress()} are responsible for managing user state and data.
		\item \textbf{State Management:} Stores states \texttt{isLoggedIn}, \texttt{username}, and \texttt{phasesCompleted}, which are used to dynamically update the user interface.
		\item \textbf{Data Management:} Connects with the \texttt{keyPressDao} database to process data. \texttt{onKeyPress} saves key press events into the database, \texttt{exportDataToTsv} exports the collected data into TSV files.
	\end{itemize}
\end{itemize}
In the app \textbf{DataStore} is used for storing login state and the user's ID. It has been implemented in \texttt{UserPreferences} class, and stores data such as:
\begin{itemize}
	\item \texttt{LOGGED\_IN\_KEY} - login state
	\item \texttt{USERNAME\_KEY} - user's ID.
\end{itemize}
\begin{figure}[H]
	\centering
	\includegraphics[width=0.8\linewidth]{images/CompanionObject.png}
	\caption{UserPreferences.kt}
	\label{fig:companion_object_view}
\end{figure}
This data is stored in the app's preferences file and can be accessed via dataStore object using:

\begin{itemize}
	\item \texttt{isLoggedIn} - returns login state as \texttt{Flow<Boolean>}
	\item \texttt{username} - returns user's ID as \texttt{Flow<String>}
	\item \texttt{setLoggedIn()} - saves login state and user's ID into DataStore
\end{itemize}

\begin{figure}[H]
	\centering
	\includegraphics[width=0.8\linewidth]{images/UPFunctions.png}
	\caption{UserPreferences.kt}
	\label{fig:user_preferences_functions_view}
\end{figure}

The use of DataStore enabled the data to be stored securely, accessed and modified easily, and it is always available, which makes it a reliable and efficient way to manage user preferences and app state.

\subsection{User Interface Design}
The application design follows a minimalistic approach to make it intuitive and easy to use for everyone. 
\begin{itemize}
	\item 
	\texttt{Login Screen} \ref{fig:login_screen} \newline
	After launching the application for the first time, the user is presented with the \texttt{Login screen}. It contains \texttt{TextInput} field for entering the university ID, which was evenly distributed among contributors to simplify testing, and the \texttt{Log in} button which stores the ID and navigates the user to the \texttt{Home Screen}.
	\item 
	\texttt{Home Screen} \ref{fig:home_screen} \newline
	The home screen displays three buttons and a simple note explaining what the user should do. The \texttt{Logout} button navigates back to the \texttt{Login Screen}, while two other buttons lead to either testing or training screens.
	\item 
	\texttt{Training Screen} \ref{fig:training_screen} \newline
	The training screen is designed for collecting data for training purposes. It includes \texttt{TextInput} field for typing user input, a \texttt{Button} to proceed to the next phase, and \texttt{Text} indicators showing how many chars are needed to complete the phase (300 each phase) and how many phases remain (5 phases in total) to complete the process of collecting training data. Additionally, there are two notes instructing the user to maintain the writing style throughout the whole process and to change the position after each phase while writing (explained in subsection~\ref{sec:data_collection}). \newline
	To ensure that typing is done in the most natural way, the default android keyboard is used. 
	\item 
	\texttt{Testing Screen} \ref{fig:testing_screen} \newline
	The testing screen includes \texttt{TextInput} field for typing the test input, a \texttt{Button} that sends the input to the server and stores it locally, and \texttt{Text} indicators showing how many characters need to be written (in this case, 100). After fulfilling the requirements, the user sends their input to the server, which evaluates it against the trained model. The server then returns feedback and the user is presented with a recognition rate percentage on a circular progress bar and a message indicating whether the model recognised them or not \ref{fig:testing_screen_example}. 
\end{itemize}


\begin{figure}[H]
	\centering
	\includegraphics[width=0.32\linewidth]{images/login_screen.png}
	\caption{Login screen}
	\label{fig:login_screen}
\end{figure}

\begin{figure}[H]
	\centering
	\includegraphics[width=0.32\linewidth]{images/home_screen.png}
	\caption{Home screen}
	\label{fig:home_screen}
\end{figure}

\begin{figure}[H]
	\centering
	\includegraphics[width=0.32\linewidth]{images/training_screen.png}
	\caption{Training screen}
	\label{fig:training_screen}
\end{figure}

\begin{figure}[H]
	\centering
	\includegraphics[width=0.32\linewidth]{images/testing_screen.png}
	\caption{Testing screen}
	\label{fig:testing_screen}
\end{figure}

\begin{figure}[H]
	\centering
	\includegraphics[width=0.32\linewidth]{images/testing_screen_example.png}
	\caption{Testing screen example}
	\label{fig:testing_screen_example}
\end{figure}

\subsection{Data Collection Process}
\label{sec:data_collection}
Data collection occurs in two stages, training and testing
\begin{itemize}
	\item 
	Training data collection begins on the \texttt{Training Screen} \ref{fig:training_screen}, where the user is asked to input meaningful sentences. The process consists of 5 phases. Each phase requires the user to type 300 characters. Once the requirement is met, the user progresses to the next phase until all 5 phases are completed (1500 characters in total). Additionally, there is a note instructing the user to maintain a consistent writing style throughout all phases. Also the user is asked to change their position after each phase while writing. This is important for accelerometer data collection (which is not used at the moment), as it helps exclude situations where the phone is lying on the table or being held in an atypical way.  
	\item 
	Testing data collection takes place on the \texttt{Testing Screen} \ref{fig:testing_screen}, where the user is required to write 100 characters, again in meaningful sentences. Once this is done, the testing phase is complete.
\end{itemize}
After each phase, the collected data is saved in a \texttt{.tsv} file, sent to the server, and stored locally in the phone's downloads directory. The \texttt{exportDataToTsv} \ref{fig:export_data_code} function from \texttt{MainViewModel.kt} handles the export of key press data. \newline Firstly it retrieves latest key press events using the \texttt{keyPressDao.getNLatestKeyPresses} method, converting the data into a \texttt{.tsv} format using the \texttt{keyPressesToTsv} function. \newline
Depending on which phase the user is in, the function determines the different type of operation to perform.
\begin{itemize}
	\item 
	If the user is in inference phase, the data will be used for inference.
	\item 
	If the user is in training phase, the data will be used for training.
	\item 
	If the number of completed phases exceeds the required amount, the function exits without performing any other action.
\end{itemize}

After processing data \texttt{saveTsvToDownloads} stores data locally, and \texttt{sendTsvToFastApi} sends data to the server (An example of the \texttt{.tsv} file containing the saved data is shown in Figure \ref{fig:tsv_example}). The username, and the relevant phase is included in the file name. \newline
This function ensures that after each phase of training and testing, the data is collected, stored, and transmitted.

\begin{figure}[H]
	\centering
	\includegraphics[width=0.8\linewidth]{images/ExportData.png}
	\caption{MainViewModel.kt}
	\label{fig:export_data_code}
\end{figure}

\begin{figure}[H]
	\centering
	\includegraphics[width=0.8\linewidth]{images/data_example.png}
	\caption{An example of the \texttt{.tsv} file containing saved data.}
	\label{fig:tsv_example}
\end{figure}

\subsection{Communication with the server}

The application communicates with the server using an HTTPS connection.
\begin{itemize}
	\item 
	\textbf{Server URL and Request Structure}
	\begin{itemize}
		\item 
		The server is accessed via an HTTPS endpoint. The base URL is defined as \newline \texttt{https://192.168.1.100:8000}.
		\item 
		The API endpoint is dynamically created with the use of a route and query parameters to the base URL. For example, the endpoint for sending data contains the username as a query parameter: \newline
		\texttt{https://192.168.1.100:8000/<api\_string>?username=<username>}
		\item 
		The data is sent using the POST method.
	\end{itemize}
	\item
	\textbf{Data format} \newline
	The data sent to the server is stored in \texttt{.tsv} (tab-separated values) file, containing headers and the detailed information about key presses.
	\item 
	\textbf{Secure Connection Setup}
	\begin{itemize}
		\item 
		The application uses \texttt{OkHttpClient} library for handling network requests.
		\item 
		A \texttt{.cert} certificate (stored in \texttt{res/raw/cert}) is used to establish a secure and trustworthy \texttt{SSL/TLS} connection.
	\end{itemize}
	\item 
	\textbf{Sending request}
	\begin{itemize}
		\item 
		Requests are executed asynchronously using the \texttt{enqueue} method.
		\item 
		If successful, the server's response is processed, and the application displays the result to the user.
		\item 
		On failure, the error is logged, and the user is notified.  
	\end{itemize}
	\item 
	\textbf{Error Handling}
	\begin{itemize}
		\item 
		Network errors (e.g., problems with connection) and server errors are logged for easier debugging.
		\item 
		A callback system is implemented to communicate feedback to the user. 
	\end{itemize}
	
	
\end{itemize}


\subsection{Potential uses}
The application analysing the typing habits of users (keystroke dynamics) has a wide range of applications in a various fields, providing an additional layer of security. It could for example be used to verify user's identity in online banking or for continuous identity checks during sessions in corporate environments. Keystroke dynamics has potential to become a successful and user-friendly security measure.


\input{chapters/04-s3-project-features}
\input{chapters/04-s4-project-model}

\section{Model fine-tuning and hyperparameters - metrics}

TODO for IW: write about the fine-tuning process, the metrics used and why are they used, cross-validations used etc. You can also post some hyperparam statistics here.

\section{Testing model on users}

TODO for xxx: when the tests are done (hopefully a week) we will discuss this. Also, there could be a part about many-users recognition.

\subsection{Cross-smartphone user validation}
TODO: what happens if two users train on smartphones that are not their own? What happens, if they cross-use their original model on another phone?

\subsection{Discussion}
TODO: discuss the findings.
\input{chapters/05-conclusion}

%--------------------------------------
% Literatura
%--------------------------------------

\bibliographystyle{plain}{\raggedright\sloppy\small\bibliography{bibliography}}

%--------------------------------------
% Dodatki
%--------------------------------------

\cleardoublepage\appendix%

%--------------------------------------
% Informacja o prawach autorskich
%--------------------------------------

\end{document}
