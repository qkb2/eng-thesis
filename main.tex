%%%%%%%%%%%%%%%%%%%%%%%%%%%%%%%%%%%%%%%%%%%%%%%%%%
%% Bachelor's & Master's Thesis Template        %%
%% Copyleft by Dawid Weiss & Marta Szachniuk    %%
%% Faculty of Computing and Telecommunication   %%
%% Poznan University of Technology, 2020        %%
%%%%%%%%%%%%%%%%%%%%%%%%%%%%%%%%%%%%%%%%%%%%%%%%%%

\documentclass[english,bachelor,a4paper,oneside]{ppfcmthesis}


\usepackage[utf8]{inputenc}
\usepackage[OT4]{fontenc}
\usepackage{graphicx}
\usepackage{hyperref}
\usepackage{float}
\usepackage{tikz}
\usepackage{amsmath}
\usetikzlibrary{shapes.geometric, arrows, backgrounds, fit, positioning}


% Define styles
\tikzstyle{startstop} = [rectangle, rounded corners, minimum width=3cm, minimum height=1cm, align=center, draw=black, fill=blue!30]
\tikzstyle{process} = [rectangle, minimum width=3cm, minimum height=1cm, align=center, draw=black, fill=blue!30]
\tikzstyle{arrow} = [thick,->,>=stealth]

% 
\usepackage{xcolor}
\newcommand\myworries[1]{\textcolor{red}{#1}}
%

%--------------------------------------
% Strona tytułowa
%--------------------------------------

% Autorzy pracy, jeśli jest ich więcej niż jeden
% wstaw między nimi separator \and
\author{%
   Jakub Grabowski \album{151825} \and 
   Filip Kozłowski \album{151823} \and 
   Krzysztof Matyla \album{151778} \and 
   Igor Warszawski \album{151585}}
\authortitle{}                                % Do not change.

\title{Biometric identification of a smartphone user using graph neural networks}

% Your supervisor comes here.
\ppsupervisor{~dr hab.~inż.~Szymon Szczęsny, ~prof. PP} 

% Year of final submission (not graduation!)
\ppyear{2025}                                 


\begin{document}

% Front matter starts here
\frontmatter\pagestyle{empty}%
\maketitle\cleardoublepage%

%--------------------------------------
% Miejsce na kartę pracy dyplomowej
%--------------------------------------

\thispagestyle{empty}\vspace*{\fill}%
\begin{center}Tutaj będzie karta pracy dyplomowej;\\oryginał wstawiamy do wersji dla archiwum PP, w pozostałych kopiach wstawiamy ksero.\end{center}%
\vfill\cleardoublepage%

%--------------------------------------
% Spis treści
%--------------------------------------

\pagenumbering{Roman}\pagestyle{ppfcmthesis}%
\tableofcontents* 
\cleardoublepage % Zaczynamy od nieparzystej strony

%--------------------------------------
% Rozdziały
%--------------------------------------

%Najwygodniej jeśli każdy rozdział znajduje się w oddzielnym pliku
\mainmatter%

\chapter{Introduction}

Biometric data is a widely used -- especially on mobile devices -- for user authentication. It is also used for person recognition. As of 2024, most smartphones have biometric sensors, such as fingerprint readers (FIX SOURCE). Many computers also provide biometric authentication via face recognition (FIX SOURCE). These are, however, not the only possible recognition or authentication methods that use biometric data.

The project aimed to develop a model, along with a corresponding mobile app, capable of recognizing the user by their biometric data contained mostly within the keystroke data. The users in the study, which was a part of the project, provided their data by entering long stretches of text as testing data. Models were created for each user, with the standard model testing procedures and validations. A subgroup of the study participants was also asked to verify the model in real-life testing by writing short paragraphs in the application, which were sent to the server for user verification.

The scope of the work was to create a mobile application capable of gathering the keystroke data, which could then be used by the server to create Graph Neural Network (GNN) models tasked with recognizing the user as opposed to other possible users. Also in the scope was performing a study on a group of participants who provided the data for the project and participated in the application and model demonstration and testing.

The sources used in this thesis mostly concerned the two following groups: studies of keystroke data models and their effectiveness and the specialist literature on the topic of Graph Neural Networks.

The thesis has the following structure:
Chapter 2 consists of some theory concerning biometrics, especially in the context of user input data, with a small literature review about using biometrics for user recognition.
Chapter 3 contains basic theoretics about Graph Convolutional Networks, which are used for user recognition in the model created for the project.
Chapter 4 is a brief overview of the project, explaining its components and the relationships between them. It includes the following sections:
Section 4.1 consists of the description of the server.
Section 4.2 describes the mobile application used for user data collection and model validation.
Section 4.3 contains a description of the Neural Network model used for user recognition, complete with the hyperparameters used in model training and validation.
Section 4.4 describes the feature selection used for a model.
Section 4.5 discusses the metrics used in the model testing on data gathered from users and the testing results.
Section 4.6 concerns the user testing with the help of study participants and the study results.
Chapter 5 is a conclusion to the thesis.

Work on this project was divided as follows:
Jakub Grabowski created the mobile application, set up and coordinated the project, and researched biometrics for his thesis paper.
Filip Kozłowski created the server and integrated the GNN model with it. He also planned and implemented communication between the server and the application.
Krzysztof Matyla helped in creating the mobile application, provided testing for various parts of the project, and coordinated user testing.
Igor Warszawski planned and implemented the GNN model used on the server. He also tested and validated the results, together with Filip Kozłowski.

\chapter{Biometrics in mobile devices - theory}

Fundamental to the goal of the project was the use of biometric data in user identification. Biometric data can be defined as measurements of some unique characteristics of an individual. These can largely be divided into two main categories: physiological data, which is the measurement of the inherent characteristics of an individual's body, such as a fingerprint, an iris scan or a face scan, and behavioral data, which measures the person's movements, behaviors, speech patterns etc. \cite{Abde2023}

Uniqueness of one's body is well known in biology. Features that may be used for person's identification are for example (FIX SOURCE: https://www.biometricsinstitute.org/what-is-biometrics/types-of-biometrics/):

\begin{enumerate}
    \item \textbf{DNA} -- found in cells of the living organisms, this acid carries genetic information.
    \item \textbf{Eye features} -- human iris, retina and scleral veins can be used in eye scans.
    \item \textbf{Face} -- full face scan is often used for user recognition, for example in mobile devices and laptops. (FIX SOURCE: https://developers.google.com/ml-kit/vision/face-detection, https://support.apple.com/en-us/102381)
    \item \textbf{Fingerprints and finger shape} -- fingerprints are widely used in forensics (FIX SOURCE: https://www.nist.gov/forensic-biometrics) and in digital scanners on mobile devices and laptops.
\end{enumerate}

Other, less popular ways of identifying a person are for example: ear shape, gait, hand shape, heartbeat, keystroke dynamics, signatures, vein scans and voice recognition.

One possible way to extract data from a person's behavior is via *keystroke dynamics*. This type of behavioral biometrics is acquired from a user by means of a keyboard or other typing device and records and extracts features from the way the keyboard is used. Most commonly used and almost universally applicable to any keyboard device is the measurement of timings between each character typed. If the user uses a physical keyboard, it is also convenient to derive the following features \cite{Shar2023}:

\begin{enumerate}
    \item \textbf{Hold Time} -- time between key press and release
    \item \textbf{Down-Down Time} -- time between first key press and second key press
    \item \textbf{Up-Up Time} -- time between first key release and second key release
    \item \textbf{Up-Down Time} -- time between first key release and second key press
    \item \textbf{Down-Up Time} -- time between first key press and second key release.
\end{enumerate}

It can therefore be said that keystroke dynamics focus mostly on identifying user's rhythmic patterns in their keystrokes. Such data can be used in conjunction with for example a password or a passphrase as a means of additional protection against password theft -- this idea was already being tested in 1990 by Joyce and Gupta (FIX SOURCE). The concept itself was already being studied in 1980, with Gaines et al. (FIX SOURCE) experimenting on secretaries' typing latency data. By 1997, clustering methods were already being used in experiments on user data on a small scale (42 profiles) by Monrose et al. \cite{Monr1997}. Algorithms used for such data evolved after that point and the raise in popularity of neural networks.

Lu et al. \cite{Lu2020} used a combined CNN+RNN approach to obtain the results of ERR of 2.36\% on Buffalo dataset and 5.97\% on Clarkson II datasets. Çeker and Upadhyaya (FIX SOURCE) used a CNN to create a multiple classification model -- this method is mostly suitable for smaller datasets with smaller number of users, as opposed to creating a personalized model for each user. This method had an ERR of 2.3\%. 

Another aspect of the keystroke dynamics recognition methods is how and when the data is collected. The systems can either work with some specific strings being typed by the user (like in Çeker and Upadhyaya case) or with the users being free to type anything within some length constraints (like in Lu 2020 \cite{Lu2020}). In this project, the second approach was chosen as the more realistic one.

With some keyboards it may be more difficult to gather all the possible features. Even basic feature, such as the hold time can prove difficult to gather when using for example GBoard on mobile devices, which does not naturally send key press and key release information to the application (SOURCE). This information can thus only be gathered in approximation or by building another virtual keyboard application. This, however, has its drawbacks. The users are generally used to one type of keyboard (on mobile it may be for example GBoard or SwiftKey), so forcing them to use another type of keyboard may be detrimental. Same person may write somewhat differently on different keyboards and machines. This study includes a small subsection on cross-smartphone compatibility of the model, for example concerning two users using each others' smartphones.

While the model may be less accurate because of the lack of features, there can be some ways to mitigate it. Some other features can be added, which are largely specific to mobile devices, such as accelerometer data, or a larger sample can be used. A few of those options were considered by the researchers, and the results will be discussed in the next chapters (???).

The keystroke identification can also rely on other data gathered from the keyboard, such as the specifics of letters used, their average frequencies, most common connections between the letter or other statistics \cite{Wang2024}. These statistics can be modeled in many ways. If the average Up-Up Time between two keys is gathered from the data, a graph can be formed, having additional features as see fit by the designers. Such graphs were constructed for the Neural Network models constructed in this study, which will be discussed in the next chapter.

According to EU guidelines, all data used in Machine Learning models should also be ethically sourced (FIX SOURCE: EU GUIDELINES \url{https://eur-lex.europa.eu/legal-content/EN/TXT/?uri=CELEX%3A32024R1689}). In this study, all data was sourced from willing participants and anonymized using unique ID numbers given to the participants by the researchers.

\chapter{Graph Convolutional Networks - theory}

Graphs can defined as mathematical structures $G$ consisting of a set of vertices $V$, a set of edges $E$ and an incidence function $\phi$, along with many variations and generalizations to such structure, can be used for describing entities, which are related to each other in some way. An example of such model could be a computer network graph or citation network. Neurons can also be modelled in a similar way. Relation data can often be best described using such graphs. \cite{Lesk2024}

Some problems relating to such data can be solved using Convolutional Neural Networks -- this can also be the case for keystroke dynamics data, such as with Lu et al. \cite{Lu2020} or Sharma et al. \cite{Shar2023}. However, it can be reasoned that the Graph Neural Networks can also perform such tasks, with connections in graph data being used more directly in the model itself.

\section{Graph Neural Networks}
TODO: how do GNN graph embeddings etc. work?

\section{Convolutional Networks and Graph Convolutional Networks}
TODO: how do GCNs and GCN networks differ? Network design is better left for project model section.

\section{Graph-level prediction in GNN}
TODO: graph-level prediction for GCN network.

\chapter{Gathering keystroke data on mobile devices}

There are many ways to recognise a phone user using biometrics, such as scanning fingerprints or facial recognition. It is very useful for security purposes. The ease of use and reliability have made passwords less popular and led to their replacement by biometrics. However, since other biometric methods are also available, it is reasonable to test if biometrics derived from writing button press intervals and phone orientation could also be a reliable way to recognise the user. To collect data and test the results, the mobile application was created.
The main goal of the application is to gather data with an easy-to-use, intuitive interface, send the data to a server for training purposes, check if the model recognises the user. 

As previously stated, State of the Art models can actually perform well (FIXSOURCE) on such data. These models are however usually trained on data gathered from physical keyboards. Additionally, the Neural Network model created for user identification was chosen to be based on Graph Convolutional Networks, which differ from models used by many researchers in the past (FIXSOURCE).
Because of that, an important part of the project was a study of results and data gathered, which is presented in chapter 3.4 and 3.5.

TODO: find statistics and add them to sources



\section{Use cases}
The main goal of this application was the identification of users based on their distinctive typing behavior, which is known as keystroke dynamics. By using machine learning models and encrypted server transmission for analyzing the collected data, the application aimed to provide an additional layer of security beyond passwords or basic biometrics. This project tried to establish whether this type of behavioral biometric can be a reliable way of user authentication. \newline
The following use cases illustrate how the user would interact with the application and its features.

\begin{itemize}
	\item \textbf{Logging into the application}
	\begin{itemize}
		\item \textbf{Purpose:} Allowing the user to log in and associate the application with a predefined \texttt{ID}.
		\item \textbf{Steps:}
		\begin{itemize}
			\item The user opens the application.
			\item On the \texttt{Login Screen}, the user enters their \texttt{ID} in the input field.
			\item The user clicks the \texttt{Login} button to proceed.
			\item The application stores given \texttt{ID} for further operations.
		\end{itemize}
		\item \textbf{Result:} The user is logged in and is redirected to the \texttt{Home Screen}.
	\end{itemize}
	
	\item \textbf{Data collection from key presses}
	\begin{itemize}
		\item \textbf{Purpose:} Storing users' keyboards interaction data for analysis.
		\item \textbf{Steps:}
		\begin{itemize}
			\item The user navigates to \texttt{Training Screen}.
			\item The user types a predefined number of characters in total throughout 5 phases to complete training.
			\item The application registers data for every key press, including:
			\begin{itemize}
				\item Key ID (e.g., \texttt{A}, \texttt{h}, \texttt{3})
				\item Timestamp of key press action
				\item Press duration
				\item Accelerator data (X, Y, Z axis)
			\end{itemize}
			\item Data is stored in \texttt{KeyPressEntity} object for further processing.
		\end{itemize}
		\item \textbf{Result:} Full key press data is saved in the application, ready to be transformed into \texttt{TSV} format, transmitted to the server, or stored locally.
	\end{itemize}
	
	\item \textbf{Testing how well the model recognizes the user}
	\begin{itemize}
		\item \textbf{Purpose:} Verifying if the user entering data is the one associated with their \texttt{ID}.
		\item \textbf{Steps:}
		\begin{itemize}
			\item The user navigates to \texttt{Testing Screen}.
			\item The user types a predefined number of characters into the input field.
			\item The application registers the key press data.
			\item The data is transformed into TSV format, stored locally, and sent to the server:
			\begin{itemize}
				\item The application ensures the connection is secured with \texttt{SSL/TLS}.
				\item The \texttt{POST} method is used to send the data.
			\end{itemize}
			\item The server processes the data using the trained model.
			\item The server sends back a response to the application, including:
			\begin{itemize}
				\item Information on whether the user was recognized.
				\item The percentage of compliance with the user.
			\end{itemize}
		\end{itemize}
		\item \textbf{Result:} The application displays the result to the user.
	\end{itemize}
	
\end{itemize}

\section{Server structure and communication with the application}

TODO for FK: technical docs for the server and connections between the app and the server. Data layer can also be touched a little.

Mobile application can communicate with the server, which can be locally hosted on a computer. The programmer needs to...

Server uses FastAPI, which is...

Server has the following endpoints, which are used by application for...

\section{Mobile application for data gathering and model testing}

The application was written for Android devices supporting Android 8.1 or newer. As of 2024 \cite{androidStats}, more than 93\% of Android devices should be compatible. The Android platform was chosen, as it was easier to test on and find a study group of the Android users as opposed to the iOS users (according to \cite{operatingSystemDistribution} , significantly more people in Poland, where the researchers are based in, use Android devices).

Technology used in the mobile application itself was Jetpack Compose, which is quoted by Google to be "Android’s recommended modern toolkit for building native UI" \cite{jetpackCompose}. Language used was Kotlin. Persisence was achieved by using Android Room, which provided an abstraction layer over SQLite database, which was used for data collection.

\subsection{Model View ViewModel and DataStore}

The application uses Model-View-ViewModel (MVVM) provided by Jetpack Compose design pattern to support a clear separation of concerns. 
\begin{itemize}
	\item \textbf{Model:} Data is modeled using \texttt{KeyPressEntity} class, which represents a single key press event. It includes: 
	\begin{itemize}
		\item \textbf{Key} (\texttt{String}): The key pressed by the user.
		\item \textbf{Press Time} (\texttt{Long}): The exact timestamp of the key press event.
		\item \textbf{Duration} (\texttt{Long}): The time elapsed since the last key press event.
		\item \textbf{Accelerometer Data} (\texttt{Float}): Not used currently but could be useful for the future development of the application.
	\end{itemize}
	\begin{figure}[H]
		\centering
		\includegraphics[width=0.8\linewidth]{images/DataModel.png}
		\caption{KeyPressEntity.kt}
		\label{fig:data_model_view}
	\end{figure}
	
	The \texttt{KeyPressEntity} is stored in a local SQLite database via Room.
	
	\item \textbf{View:} The user interface is implemented using \textbf{Jetpack Compose}, a declarative UI framework. Key components of the view contain:
	\begin{itemize}
		\item \textbf{Input Fields:} Lets users enter their credentials (University ID) and use the application for training or testing by pressing keys. 
		\item \textbf{Completion progress:} Informs users on what phase they are and displays progress of completion, linked to the \texttt{phasesCompleted} state in the \texttt{MainViewModel}.
		\item \textbf{Buttons:} Used for logging in, logging out, jumping phases and sending or downloading the data collected through training or testing stage.
	\end{itemize}
	\item \textbf{ViewModel:} This role is fulfilled by \texttt{MainViewModel}, which manages the application logic, handles interactions between the model and the view, and maintains the state of the app. \newline
	The \texttt{MainViewModel} class manages this operations through:
	\begin{itemize}
		\item \textbf{Logic Handling:} Methods such as \texttt{login()}, \texttt{logout()}, \texttt{clearDatabase()}, and \texttt{onKeyPress()} are responsible for managing user state and data.
		\item \textbf{State Management:} Stores states \texttt{isLoggedIn}, \texttt{username}, and \texttt{phasesCompleted}, which are used to dynamically update the user interface.
		\item \textbf{Data Management:} Connects with the \texttt{keyPressDao} database to process data. \texttt{onKeyPress} saves key press events into the database, \texttt{exportDataToTsv} exports the collected data into TSV files.
	\end{itemize}
\end{itemize}
In the app \textbf{DataStore} is used for storing login state and the user's ID. It has been implemented in \texttt{UserPreferences} class, and stores data such as:
\begin{itemize}
	\item \texttt{LOGGED\_IN\_KEY} - login state
	\item \texttt{USERNAME\_KEY} - user's ID.
\end{itemize}
\begin{figure}[H]
	\centering
	\includegraphics[width=0.8\linewidth]{images/CompanionObject.png}
	\caption{UserPreferences.kt}
	\label{fig:companion_object_view}
\end{figure}
This data is stored in the app's preferences file and can be accessed via dataStore object using:

\begin{itemize}
	\item \texttt{isLoggedIn} - returns login state as \texttt{Flow<Boolean>}
	\item \texttt{username} - returns user's ID as \texttt{Flow<String>}
	\item \texttt{setLoggedIn()} - saves login state and user's ID into DataStore
\end{itemize}

\begin{figure}[H]
	\centering
	\includegraphics[width=0.8\linewidth]{images/UPFunctions.png}
	\caption{UserPreferences.kt}
	\label{fig:user_preferences_functions_view}
\end{figure}

The use of DataStore enabled the data to be stored securely, accessed and modified easily, and it is always available, which makes it a reliable and efficient way to manage user preferences and app state.

\subsection{User Interface Design}
The application design follows a minimalistic approach to make it intuitive and easy to use for everyone. 
\begin{itemize}
	\item 
	\texttt{Login Screen} \ref{fig:login_screen} \newline
	After launching the application for the first time, the user is presented with the \texttt{Login screen}. It contains \texttt{TextInput} field for entering the university ID, which was evenly distributed among contributors to simplify testing, and the \texttt{Log in} button which stores the ID and navigates the user to the \texttt{Home Screen}.
	\item 
	\texttt{Home Screen} \ref{fig:home_screen} \newline
	The home screen displays three buttons and a simple note explaining what the user should do. The \texttt{Logout} button navigates back to the \texttt{Login Screen}, while two other buttons lead to either testing or training screens.
	\item 
	\texttt{Training Screen} \ref{fig:training_screen} \newline
	The training screen is designed for collecting data for training purposes. It includes \texttt{TextInput} field for typing user input, a \texttt{Button} to proceed to the next phase, and \texttt{Text} indicators showing how many chars are needed to complete the phase (300 each phase) and how many phases remain (5 phases in total) to complete the process of collecting training data. Additionally, there are two notes instructing the user to maintain the writing style throughout the whole process and to change the position after each phase while writing (explained in subsection~\ref{sec:data_collection}). \newline
	To ensure that typing is done in the most natural way, the default android keyboard is used. 
	\item 
	\texttt{Testing Screen} \ref{fig:testing_screen} \newline
	The testing screen includes \texttt{TextInput} field for typing the test input, a \texttt{Button} that sends the input to the server and stores it locally, and \texttt{Text} indicators showing how many characters need to be written (in this case, 100). After fulfilling the requirements, the user sends their input to the server, which evaluates it against the trained model. The server then returns feedback and the user is presented with a recognition rate percentage on a circular progress bar and a message indicating whether the model recognised them or not \ref{fig:testing_screen_example}. 
\end{itemize}


\begin{figure}[H]
	\centering
	\includegraphics[width=0.32\linewidth]{images/login_screen.png}
	\caption{Login screen}
	\label{fig:login_screen}
\end{figure}

\begin{figure}[H]
	\centering
	\includegraphics[width=0.32\linewidth]{images/home_screen.png}
	\caption{Home screen}
	\label{fig:home_screen}
\end{figure}

\begin{figure}[H]
	\centering
	\includegraphics[width=0.32\linewidth]{images/training_screen.png}
	\caption{Training screen}
	\label{fig:training_screen}
\end{figure}

\begin{figure}[H]
	\centering
	\includegraphics[width=0.32\linewidth]{images/testing_screen.png}
	\caption{Testing screen}
	\label{fig:testing_screen}
\end{figure}

\begin{figure}[H]
	\centering
	\includegraphics[width=0.32\linewidth]{images/testing_screen_example.png}
	\caption{Testing screen example}
	\label{fig:testing_screen_example}
\end{figure}

\subsection{Data Collection Process}
\label{sec:data_collection}
Data collection occurs in two stages, training and testing
\begin{itemize}
	\item 
	Training data collection begins on the \texttt{Training Screen} \ref{fig:training_screen}, where the user is asked to input meaningful sentences. The process consists of 5 phases. Each phase requires the user to type 300 characters. Once the requirement is met, the user progresses to the next phase until all 5 phases are completed (1500 characters in total). Additionally, there is a note instructing the user to maintain a consistent writing style throughout all phases. Also the user is asked to change their position after each phase while writing. This is important for accelerometer data collection (which is not used at the moment), as it helps exclude situations where the phone is lying on the table or being held in an atypical way.  
	\item 
	Testing data collection takes place on the \texttt{Testing Screen} \ref{fig:testing_screen}, where the user is required to write 100 characters, again in meaningful sentences. Once this is done, the testing phase is complete.
\end{itemize}
After each phase, the collected data is saved in a \texttt{.tsv} file, sent to the server, and stored locally in the phone's downloads directory. The \texttt{exportDataToTsv} \ref{fig:export_data_code} function from \texttt{MainViewModel.kt} handles the export of key press data. \newline Firstly it retrieves latest key press events using the \texttt{keyPressDao.getNLatestKeyPresses} method, converting the data into a \texttt{.tsv} format using the \texttt{keyPressesToTsv} function. \newline
Depending on which phase the user is in, the function determines the different type of operation to perform.
\begin{itemize}
	\item 
	If the user is in inference phase, the data will be used for inference.
	\item 
	If the user is in training phase, the data will be used for training.
	\item 
	If the number of completed phases exceeds the required amount, the function exits without performing any other action.
\end{itemize}

After processing data \texttt{saveTsvToDownloads} stores data locally, and \texttt{sendTsvToFastApi} sends data to the server (An example of the \texttt{.tsv} file containing the saved data is shown in Figure \ref{fig:tsv_example}). The username, and the relevant phase is included in the file name. \newline
This function ensures that after each phase of training and testing, the data is collected, stored, and transmitted.

\begin{figure}[H]
	\centering
	\includegraphics[width=0.8\linewidth]{images/ExportData.png}
	\caption{MainViewModel.kt}
	\label{fig:export_data_code}
\end{figure}

\begin{figure}[H]
	\centering
	\includegraphics[width=0.8\linewidth]{images/data_example.png}
	\caption{An example of the \texttt{.tsv} file containing saved data.}
	\label{fig:tsv_example}
\end{figure}

\subsection{Communication with the server}

The application communicates with the server using an HTTPS connection.
\begin{itemize}
	\item 
	\textbf{Server URL and Request Structure}
	\begin{itemize}
		\item 
		The server is accessed via an HTTPS endpoint. The base URL is defined as \newline \texttt{https://192.168.1.100:8000}.
		\item 
		The API endpoint is dynamically created with the use of a route and query parameters to the base URL. For example, the endpoint for sending data contains the username as a query parameter: \newline
		\texttt{https://192.168.1.100:8000/<api\_string>?username=<username>}
		\item 
		The data is sent using the POST method.
	\end{itemize}
	\item
	\textbf{Data format} \newline
	The data sent to the server is stored in \texttt{.tsv} (tab-separated values) file, containing headers and the detailed information about key presses.
	\item 
	\textbf{Secure Connection Setup}
	\begin{itemize}
		\item 
		The application uses \texttt{OkHttpClient} library for handling network requests.
		\item 
		A \texttt{.cert} certificate (stored in \texttt{res/raw/cert}) is used to establish a secure and trustworthy \texttt{SSL/TLS} connection.
	\end{itemize}
	\item 
	\textbf{Sending request}
	\begin{itemize}
		\item 
		Requests are executed asynchronously using the \texttt{enqueue} method.
		\item 
		If successful, the server's response is processed, and the application displays the result to the user.
		\item 
		On failure, the error is logged, and the user is notified.  
	\end{itemize}
	\item 
	\textbf{Error Handling}
	\begin{itemize}
		\item 
		Network errors (e.g., problems with connection) and server errors are logged for easier debugging.
		\item 
		A callback system is implemented to communicate feedback to the user. 
	\end{itemize}
	
	
\end{itemize}


\subsection{Potential uses}
The application analysing the typing habits of users (keystroke dynamics) has a wide range of applications in a various fields, providing an additional layer of security. It could for example be used to verify user's identity in online banking or for continuous identity checks during sessions in corporate environments. Keystroke dynamics has potential to become a successful and user-friendly security measure.


\chapter{GCN Model}

In modern Machine Learning, a popular type of Neural Network is a Convolutional Neural Network. Such networks generally operate on grids. A Convolutional Neural Network (CNN) has a fixed node ordering -- some input must firstly be mapped into a grid to be used with a CNN. There are ways to map many types of data into such format. Examples of researchers using CNNs for keystroke dynamics data include Sharma et al. \cite{Shar2023} or Lu et al. \cite{Lu2020}. In Lu et al. this involved applying the convolution layers over feature vectors, which were constructed in the following manner: for pairs of keys pressed in succesion in the sequence, a feature vector is created with fields: ID of first key, ID of second key, hold duration of first key, hold duration of second key, DD time (time between first press and second press) and DU time (time between first press and second release). After applying the CNN layer, GRU layers were used.

In this project, the goal was to use the graph networks that can naturally arise from keystroke data to -- on a graph level -- try to infere the users identity. The main difference was that the use of any keyboard already installed on the user's mobile phone causes some problems with gathering keystroke temporal data, as mentioned in the second chapter. Because of that, this project used mappings involving only key ...

Moreover, this project focused on creating an collection of models, one model for each user that performs binary classification rather than one large model for multiclass classification. 
This decision was made for several reasons. Firstly, such scheme allows for models to be trained on demand, as soon as a new user provides all the training data to the mobile application. Secondly, new users do not force the whole model to be retrained, 
as only one new model needs to be created, and provided all models have learned their target users sufficiently, they would be able to reject such new users without further tuning. Lastly, the one user per model scheme allows for inference to take place locally, on the target user's device. This would remove the need for remote communication with the server, thus increasing the mobile application's reliability and security. On device inference is an area of active research, such as \myworries{TODO citation needed}, the complexity of such solution was deemed to great and outside the scope of this project.


\section{Choosing features for Neural Network model}
\myworries{TODO Some introdutions}


\subsection{Data exploration}
\myworries{All data analisys on the input goes here}

\subsection{Graph creation and feature encoding}
The input for graph creation consists of two main parts, the duration of time between individual key presses, and the character of the key that was pressed.
A natural way to represent such input was to map each unique character in the input sequence to a node in the graph. 
Dierected edges were added between nodes that represent charactes apearing after each other in the input sequence. 
\myworries{TODO: add example graph visualization here}
Each time the same pair of characters appears in the input text it maps to the same egde. For each such pair, the duration is added to a list atributes for that egde, which will be aggreagted in later stages, to a form suitable for the GCN model. It would also be possible to model such pairs using a multiegde graph, as such models have been shown to perform well in other domains \myworries{TODO zacytuj "multi-edge graph for convolutional networks for power systems}. However we did not consider this approach. 

\myworries{Citation needed - some RNN paper that sequences of keys are important}\\

Having chosen this graph structure, there are many ways to encode the input data in a form suitable to an GCN.
Firstly, the edge atributes, need to be converted into node features. We found two ways to encode aggregate this information into node features.\\  
For each node $i$:
\begin{enumerate}
	\item Two values representing the average duration before and after the key represented by $i$ was pressed.
	\item Add two-dimentional vector of values, of size [number of allowed characters, 2]. Each key that can be found in the input is assinged a number. The $n$'th row in the vector corresponds to a node, with a key assined the number $n$, now called node $j$. The $n$'th row contains two values: the average duration on the edge from $i$ to $j$ and the average duration on the egde from $j$ to $i$. The values for which egdes do not exist were assigned 0.
\end{enumerate}
The clear difference between these two approaches is the level of aggregation. Method $1$ aggregates all the egde information into 2 values, while method $2$ aggregates it into a vector of values, although it imposes some limitations, such as asigning each key a unique index into this input vector. Furthermore method $2$ increases the overall size of the input data and complexity of the model.

Similarly, there is more than one way to encode key information into node features.\\
We considered three methods:
\begin{enumerate}
	\item One hot encoding of each key 
	\item One hot encoding + some symbols map to one value, numbers to one value
	\item One hot encoding vector of uncased letters without diacritical marks, special keys, one  value for symbols, one value for numbers. Two additional bits, one for capitalized letters, one for letters with diacritical marks.
\end{enumerate}

\myworries{Citation needed: That paper that said node id's are nice}.
Again as before, these methods differ by degree of agreagtion. Methods $2$ and $3$ perform some sort of compression, mapping multiple characters to the same values, while method $1$ provides a unique, one hot encoded identifier for each possible node. \myworries{Cication} found that such encoding helped the model to learn certain structures in the data. However, \myworries{Slajdy z stanfordu ze tylko jak jest ograniczona liczba liter} notes, that providing node identifiers as input features work well only for a small and known set of possible input nodes. While this requirement appears to hold true for this specific task, we found this is not the case. Many letters, which appear to be common do not apear in our dataset. \myworries{TODO Which chars never appear}. This means that the behaviour of the models would be unpredicitble for nodes with such identifiers.
Furhter more, some charaters, for example \myworries{EXMAPLEEEE like '('}, apear only once, leading models to overfit and generalise poorly.



\myworries{ADD histogram of how many times each character appears}\\
\myworries{ADD Which chars never appear}


\subsection{Feature selection - accelerometer data}
To improve the possible performance of the model, and to make futher use of the capabilities of the moblie platform, we considered using accelerometer data as a input feature for the model.
This portion of the input data comprised of three values, a measurement of the acceleration in the x, y and z plane at the moment a keystroke was registered. 
These values were aggreaged as an average for each node. Although these models performed well during training, quickly reaching low loss values, they failed to generalize, performing worse on 
validation and test datasets. 

\myworries{TODO, some accelerometer data here}
For this reason, acceleremoter data was not used for training and evaluating models discussed later.




\section{Graph Convolutional Network for user recognition}


\myworries{IW: Przesunąlem to na góre, wydaje mi sie ze to tutaj nie miało sensu
moze mozemy tutaj dodac jakas teorie o GCN konkretnie}
\section{Metrics}
\myworries{TODO: theoretical description of metrics and why we use the ones that we use}

\section{Architecture}


\subsection{Training and fine tuning}


\subsection{Data division}
During training, the models were evaluated using 5-fold cross validation.
For testing purposes, users were asked to provide an extra 100 character text. 



TODO for IW: write about the fine-tuning process, the metrics used and why are they used, cross-validations used etc. You can also post some hyperparam statistics here.

\chapter{Results}
Evaluating the performance of machine learning models is crucial for understanding their effectiveness and limitations in practical applications. In this study, the developed models were tested on user data to assess their accuracy and generalization capabilities. Various factors, such as feature representation, input sequence length, and class balance, can significantly influence model performance.

\section{Model performance on user data}
This section focuses on the results of testing the collection of models that achieved the best performance during validation. It aims to show how this solution performs on the unseen test dataset and highlights the large differences between individual models within the collection of models.

\subsection{Methodology}
The results below were calculated as an average of each model's performance on sequences of characters of the same length as those on which the model was trained.
The results presented below were obtained using the combination of input features listed in table \ref{table:hyperparams}, which yielded the best performance during model validation. The impact of each hyperparameter choice is discussed separately in dedicated subsections and compared using the testing provided by users. It is important to note that due to the large number of possible configurations, as well as the limited computing power that was available, not all combinations were tested.

\begin{center}
	\begin{table}[H]
		
\begin{center}
	\begin{tabular}{ |c|c|} 
		\hline
		Hyperparameter & Value \\
		\hline
		Input length & \textit{40} \\ 
		\hline
		Edge encoding method & \textit{Two values per node} \\		
		\hline 
		Character encoding method & \textit{Normalized Base Letter Encoding} \\		 
		\hline
		Decision threshold & \textit{0.8} \\
		\hline
	\end{tabular}
\end{center}
	\caption{Input feature choices.}
	\label{table:hyperparams}
	\end{table}
\end{center}


\subsection{FAR and FRR scores}
The table below presents the false authentication rate and false acceptance rate for all models. The motivation for using these metrics was discussed in subsection \ref{FAR_FRR_theory}.

\begin{center}
\begin{table}[H]
\begin{center}
	\begin{tabular}{ |c|c|c| } 
		\hline
		User ID & False Acceptance Rate & False Rejection Rate \\
		\hline
		21 & 0.242 & 0.119 \\
		\hline
		22 & 0.063 & 0.000 \\
		\hline
		23 & 0.200 & 0.479 \\
		\hline
		24 & 0.042 & 0.658 \\
		\hline
		25 & 0.050 & 0.015 \\
		\hline
		26 & 0.086 & 0.032 \\
		\hline
		40 & 0.080 & 0.459 \\
		\hline
		41 & 0.057 & 0.355 \\
		\hline
		42 & 0.135 & 0.379 \\
		\hline
		60 & 0.083 & 0.608 \\
		\hline
		61 & 0.066 & 0.065 \\
		\hline
		62 & 0.044 & 0.032 \\
		\hline
		81 & 0.086 & 0.802 \\
		\hline
		82 & 0.045 & 0.024 \\
		\hline
		83 & 0.116 & 0.322 \\
		\hline
		85 & 0.128 & 0.144 \\
		\hline
		86 & 0.037 & 0.000 \\
		\hline
		Average & 0.092 & 0.264 \\
		\hline
	\end{tabular}
\end{center}
\caption{FAR and FRR scores for all models.}
\label{table:FAR_FRR_base}
\end{table}
\end{center}

The same data can be visualized as a plot of the false acceptance rate versus the false rejection rate, as shown in figure \ref{fig:frr_vs_far_all_models_base}. 

\begin{figure}[H]
	\centering
	\includegraphics[width=\textwidth]{images/far_vs_frr.pdf} 	
	\caption{False acceptance rate versus False rejection rate.}
	\label{fig:frr_vs_far_all_models_base}
\end{figure}

Figure \ref{fig:frr_vs_far_all_models_base} demonstrates a large discrepancy between the models. At the same threshold level, some models were able to achieve a FAR and FRR score below 10\%, while others failed to recognize more than half of the examples in the positive class.

This first group consists of models for users 86, 62, 82, 25, 26, and 61. These models were able to effectively learn and generalize the test dataset well.
The other major group of models was those with an unacceptably high FAR exceeding 45\% and a below-average FAR score. This group includes models for users 81, 24, 60, 40, and 23. The reason for such high FAR could be twofold, either the models simply failed to generalize to unseen positive examples, or they did but with a lower confidence level than the established threshold. Other models have varying performance and are difficult to categorize into coherent groups.

\subsection{Equal Error Rate} \label{subsec_eer}
To calculate the equal error rate, the decision threshold needs to be adjusted, which can be performed globally, for all models or individually at a per--model level. Both methods of calculating this metric were analyzed.

The global EER was calculated to illustrate the performance of the entire collection with a single value:
\begin{itemize}
	\item Equal Error Rate: 0.157
	\item Threshold: 0.20
\end{itemize}
A low decision threshold was necessary to achieve an equal error rate. This suggests that most models are unable to recognize positive examples with a high level of confidence. Such a low value would not be practical in an applied setting and within the context of the system developed in this project. 

The per-model EER can be used to examine the differences among the models.
It is calculated by finding the equal error rate decision threshold for each model separately.
The results are shown in table \ref{table:EER_separate}.
\begin{center}
	\begin{table}[H]
		\begin{center}
			\begin{tabular}{ |c|c|c| } 
				\hline
				User ID & Equal Error Rate & Decision threshold \\
				\hline
				21 & 0.124 & 0.95 \\
				\hline
				22 & 0.026 & 0.95 \\
				\hline
				23 & 0.245 & 0.40 \\
				\hline
				24 & 0.270 & 0.05 \\
				\hline
				25 & 0.037 & 0.85 \\
				\hline
				26 & 0.059 & 0.80 \\
				\hline
				40 & 0.200 & 0.05 \\
				\hline
				41 & 0.081 & 0.35 \\
				\hline
				42 & 0.163 & 0.45 \\
				\hline
				60 & 0.176 & 0.05 \\
				\hline
				61 & 0.066 & 0.80 \\
				\hline
				62 & 0.038 & 0.85 \\
				\hline
				81 & 0.392 & 0.05 \\
				\hline
				82 & 0.026 & 0.95 \\
				\hline
				83 & 0.220 & 0.05 \\
				\hline
				85 & 0.131 & 0.75 \\
				\hline
				86 & 0.028 & 0.90 \\
				\hline
				Average & 0.133 & -- \\
				\hline
			\end{tabular}
		\end{center}
		\caption{Per model equal error rate.}
		\label{table:EER_separate}
	\end{table}
\end{center}

Table \ref{table:EER_separate} illustrates the differences in the models' ability to generalize unseen positive examples. This is further demonstrated by examining the changes in FAR and FRR scores with respect to the decision threshold, which is shown in figure \ref{fig:far_ffr_all_thresholds}. 

\begin{figure}[H]
	\centering
	\includegraphics[width=.8\textwidth]{images/far_frr_curves_all_models_subplots.pdf} % Replace 'example.pdf' with your PDF file name
	\caption{Change of FAR and FRR score with respect to decision threshold.}
	\label{fig:far_ffr_all_thresholds}
\end{figure}

These plots show the relationship between FAR, FRR, and the decision threshold in more detail. The point at which the two lines cross is the EER. It is worth noting the large variation in the threshold at which the EER is achieved. Models for users 21, 22, 25, 26, 61, 62, 82, 85, and 86 reach an equal error rate at or above a threshold of 0.8, which indicates a high degree of confidence in the prediction of positive class. These models have a lower EER than models with lower threshold values. To illustrate this, table \ref{table:EER_separate}, sorted by EER, is given below.

\begin{center}
	\begin{table}[H]
		\begin{center}
			\begin{tabular}{ |c|c|c| } 
				\hline
				User ID & Equal Error Rate & Decision threshold \\
				\hline
				22 & 0.026 & 0.95 \\
				\hline
				82 & 0.026 & 0.95 \\
				\hline
				86 & 0.028 & 0.9 \\
				\hline
				25 & 0.037 & 0.85 \\
				\hline
				62 & 0.038 & 0.85 \\
				\hline
				26 & 0.059 & 0.8 \\
				\hline
				61 & 0.066 & 0.8 \\
				\hline
				41 & 0.081 & 0.35 \\
				\hline
				21 & 0.124 & 0.95 \\
				\hline
				85 & 0.131 & 0.75 \\
				\hline
				42 & 0.163 & 0.45 \\
				\hline
				60 & 0.176 & 0.05 \\
				\hline
				40 & 0.2 & 0.05 \\
				\hline
				83 & 0.22 & 0.05 \\
				\hline
				23 & 0.245 & 0.4 \\
				\hline
				24 & 0.27 & 0.05 \\
				\hline
				81 & 0.392 & 0.05 \\
				\hline
			\end{tabular}
		\end{center}
		\caption{Sorted per model equal error rate.}
		\label{table:EER_separate_sorted}
	\end{table}
\end{center}

An outlier in this trend is model 41, which achieves an equal error rate of 8.1\% at a 0.35 decision threshold, outperforming models with much higher decision thresholds. Outliers like this suggest that a better approach than choosing one decision threshold could be to determine them on a per-model basis. Such a threshold could be chosen once, on some portion of the user training data through cross-validation, although such an approach would lengthen the training process. Alternatively, such a value could be adjusted dynamically, based on how often a user fails to be authenticated using the model. Such considerations could be an area of further research as they fall outside the scope of this project.
Additionally, because a global decision results in poor performance for otherwise well-performing models, it can be concluded that the average per-model EER is a better metric to use than the global EER and, as such, it will be used in the rest of this chapter.
Another interesting example is model 21, for which the FRR is almost constant, at a value of around 12\%. This indicates that all examples are recognized with a high degree of confidence, except for those 12\%, which may indicate that the user changes writing styles for part of the final input or that the training data failed to capture some characteristic of the writing style.

\subsection{Confusion matrix for all users}
To evaluate each model's performance, each model was tested using the input of every user. The results of this test are shown in figure \ref{fig:all_models_5_len40}. The values in this matrix represent the percentage of examples classified as positive. It is important to note that the values in this matrix have different interpretations depending on their position. The values on the diagonal represent the percentage of correctly classified positive examples (True Positives) -- the recall of the model. In an ideal classifier, these values would equal 100. The values outside the diagonal represent the percentage of incorrectly classified negative examples (False Positives). In an ideal classifier, these values would equal 0.

\begin{figure}[H]
	\centering
	\includegraphics[width=\textwidth]{images/all_models_5_len40.pdf}
	\caption{Matrix showing the percentage of examples classified as positive for all model--user pairs.}
	\label{fig:all_models_5_len40}
\end{figure}

Figure \ref{fig:all_models_5_len40} illustrates the different types of errors made by individual models. Specifically, it highlights which users share similarities, leading to confusion in model predictions. For example, models for users 25 and 86 perform very well for all inputs except for each other. However, this is not true for all models, as the model for user 61 makes mistakes when classifying the input for user 82. The inverse is not observed, as the model for user 82 does not misclassify user 61's data at a higher rate than others.

\section{Input features}
This section compares the impact of input encodings on the final model performance, as measured by the average FAR and FRR, as well as the average per-model EER value. The impact of these changes was measured by changing only a single hyperparameter, while all others remained the same, as shown in table \ref{table:hyperparams}. The results shown below compare performance on the testing dataset, allowing these values can be contrasted with what was discussed in the previous section. As mentioned previously, this is not how these features were selected.

\subsection{Input sequence length}
The theoretical impact of input length, as discussed in the section on graph creation, suggests that an optimal sequence length exists. Making the sequence shorter would result in graphs that do not have enough structure, for example, a chain of nodes or not enough edge data to make the correct prediction possible. Making the sequence longer would result in a graph that is too dense, or it would cause the averages calculated per node to become too aggregated.


\begin{center}
	\begin{table}[H]
		\begin{center}
			\begin{tabular}{ |c|c|c|c| } 
				\hline
				Input sequence length & FAR & FRR & EER \\
				\hline
				30 & 0.101 & 0.332 & 0.180 \\
				\hline
				40 & 0.092 & 0.264 & 0.133 \\
				\hline
				50 & 0.098 & 0.232 & 0.135 \\
				\hline
				60 & 0.079 & 0.307 & 0.158 \\
				\hline
				70 & 0.083 & 0.391 & 0.176 \\
				\hline
			\end{tabular}
		\end{center}
		\caption{Comparison of performance with different lengths of input sequence.}
		\label{table:len_vs_perf}
	\end{table}
\end{center}


Table \ref{table:len_vs_perf} partially supports the theoretical expectation, as models trained on input sequences of lengths 30, 60, and 70 achieve much higher EER scores than those trained with lengths of 40 and 50. There is a very small difference in the performance of these model collections on the aggregated metrics. 
With the difference in the EER scores being this low, it is unclear which collection of models performed better on the testing dataset.
However, interesting differences appear when compared on a per-model basis, as the changes in models were not uniform. The impact of the three selected models is shown in figure \ref{fig:selected_frr_vs_far}. 

\begin{figure}[H]
	\centering
	\subfloat[Plots for input of length 40]{
		\includegraphics[width=\textwidth]{images/selected_far_vs_frr_len40.pdf}
	}

	\subfloat[Plots for input of length 50]{
		\includegraphics[width=\textwidth]{images/selected_far_vs_frr_len50.pdf}
	}
	\caption{Comparison of selected FAR and FRR plots for different input lengths}
	\label{fig:selected_frr_vs_far}
\end{figure}

The model for user 21 shows a clear improvement as the unrecognized examples discussed in subsection \ref{subsec_eer} are now correctly classified. Conversely, the model for user 26 now classifies its positive examples with less confidence and showed an increase in EER from 5.9\% to 10.2\%. For model 85 the change in length did not impact the EER. However, it moved the decision threshold significantly.
What these comparisons demonstrate is that it is hard to reason about model performance across different input sequence lengths and a small change has a large impact even on a per-model basis.

\subsection{Character encoding}
The results of comparing methods of encoding character information in nodes are shown in the Table
\ref{table:char_encoding}.

\begin{center}
	\begin{table}[H]
		\begin{center}
			\begin{tabular}{ |c|c|c|c| } 
				\hline
				Character encoding method & FAR & FRR & EER \\
				\hline
				\textit{Normalized Base Letter Encoding} & 0.092 & 0.264 & 0.133 \\
				\hline
				\textit{Compact Alphabet Encoding} & 0.082 & 0.339 & 0.161 \\
				\hline
				\textit{Full One-Hot Encoding} & 0.070 & 0.397 & 0.197 \\
				\hline
			\end{tabular}
		\end{center}
		\caption{Comparison of performance with different edge data encoding methods.}
		\label{table:char_encoding}
	\end{table}
\end{center}

Table \ref{table:char_encoding} shows that more aggregative methods outperform those that are less compact. The lower FAR might suggest that these models try to capture more complex features and thus overfit the data.

\subsection{Edge data encoding}

\begin{center}
	\begin{table}[H]
		\begin{center}
			\begin{tabular}{ |c|c|c|c| } 
				\hline
				Edge encoding method & FAR & FRR & EER \\
				\hline
				\textit{Two dimensional vector of values per node} & 0.069 & 0.666 & 0.395 \\
				\hline
				\textit{Two values per node} & 0.092 & 0.264 & 0.133 \\
				\hline
			\end{tabular}
		\end{center}
		\caption{Comparison of performance with different edge data encoding methods.}
		\label{table:egde_encoding_comp}
	\end{table}
\end{center}

Table \ref{table:egde_encoding_comp} shows the superiority of the \textit{Two values per node} encoding method. While the average FAR remains comparable, FRR, and thus the EER, differ by a large amount. This might be caused by the same reasons as the difference between character encoding methods. The method with less aggregation performs worse, as the model might not have seen all possible two-letter combinations and failed to generalize.\\


\section{Class imbalance}
Another aspect of experimentation was class balance in the training data. The baseline model was trained on a dataset with twice as many positive examples as negative examples. A bigger proportion of positive examples was selected experimentally, as it resulted in models that performed better in recognizing the positive class.
 
The negative examples were sampled uniformly from all other users, with an offset such that the negative examples for a user were taken from the whole input text. This approach is effective for the number of users in the current dataset. However, as the number of users grows, and the length of the input sequence stays fixed, at some point each model needs to be trained only on a subset of the negative class input texts. At this point, some models may fail to generalize to unseen negative examples. 
While solving such a problem lies outside the scope of this project, the impact of using more negative examples in the training process was measured and the results are presented in table \ref{table:egde_encoding_comp}.


\begin{center}
	\begin{table}[H]
		\begin{center}
			\begin{tabular}{ |c|c|c|c| } 
				\hline
				Positive to Negative Ratio & FAR & FRR & EER \\
				\hline
				\textit{2:1} & 0.092 & 0.264 & 0.133 \\
				\hline
				\textit{1:1} & 0.080 & 0.340 & 0.177 \\
				\hline
				\textit{1:2} & 0.033 & 0.506 & 0.220 \\
				\hline
			\end{tabular}
		\end{center}
		\caption{Impact of positive to negative ratio on model collection performance.}
		\label{table:egde_encoding_comp}
	\end{table}
\end{center}

A higher percentage of negative examples were classified correctly, as indicated by the lower FAR; however, this came at the expense of a significant drop in the FRR, with almost half of the positive examples being misclassified in the case of \textit{1:2} class balance. This means that the collection of models generalized poorly to unseen positive examples, to a degree that would be unacceptable. This can also be seen when comparing the average recall scores across these two collections of models, which dropped from \textit{0.7357} to \textit{0.692} and \textit{0.494} when the number of negative examples was doubled. 


\section{Number of users}
The models developed in this project were tested on a relatively small number of users. This limited dataset poses challenges in evaluating the scalability of the solution to a larger population, which is a critical factor for deploying such a system in real-world applications. Two potential issues were identified that could impact performance as the number of users increases:
\begin{enumerate}
	\item Limited negative class examples: with a larger user base, each model would be exposed to fewer negative class examples during training. This reduction could lead to higher false acceptance rates, as the models may struggle to distinguish between the positive and negative classes effectively.  However, this issue seems less concerning since the current models tend to struggle with specific individual users rather than misclassifying negative examples uniformly.
	
	\item Similarity in writing characteristics: as the number of users grows, the probability of individuals sharing similar writing patterns increases. This overlap could lead to confusion for the models, resulting in misclassifications. Evidence of this issue was already observed on a smaller scale, with users 25 and 86, as illustrated in figure \ref{fig:all_models_5_len40}.
\end{enumerate}

To further explore these scalability concerns, experiments were conducted on two user subsets:
\begin{enumerate}
	\item Subset A: users 21, 22, 26, 60, 82, 83, 85, and 86.
	\item Subset B: users from subset A, with the addition of users 23, 24, and 62.
\end{enumerate}

Additionally, a targeted experiment was performed to determine whether users 25 and 86 could be reliably differentiated when models were trained exclusively on their datasets.

\subsection{Subset test}
The global performance of a subset of models is highly dependent on the users included in the subset. For instance, removing users 23 or 24 from the subset would likely improve overall performance, as the models for these users exhibit poor results. Conversely, removing users 22 or 26 would reduce performance, as these users consistently achieve strong results regardless of the subset composition.

To account for these variations, the models were evaluated on an individual basis. The changes in Equal Error Rate (EER) are illustrated in figure \ref{fig:subset_plot}.

\begin{figure}[H]
	\centering
	\includegraphics[width=.7\textwidth]{images/subset_test_res.png}
	\caption{"Impact of smaller user subset on EER across 8 models.}
	\label{fig:subset_plot}
\end{figure}

As demonstrated in the figure, the impact of increasing the user subset size varies significantly across models. Notably, for three of the eight tested models, the relationship was non-monotonic. Consequently, drawing definitive conclusions from this test is challenging due to the limited number of data points.

\subsection{Two-model test}
This small-scale test can be considered an extreme version of the subset test, reduced to include only two models. The purpose of this test was to evaluate whether the proposed solution could reliably differentiate between these two users under optimal conditions.

The results of this test are presented in the figure below.

\begin{figure}[H]
	\centering
	\begin{subfigure}{.5\textwidth}
		\centering
		\includegraphics[width=.99\textwidth]{images/confusion_matrix_25.pdf}
		\caption{Confusion matrix for model 25.}
		\label{fig:sub1}
	\end{subfigure}%
	\begin{subfigure}{.5\textwidth}
		\centering
		\includegraphics[width=.99\textwidth]{images/confusion_matrix_86.pdf}
		\caption{Confusion matrix for model 86.}
		\label{fig:sub2}
	\end{subfigure}
	\caption{Confusion matrices for models 25 and 86 tested exclusively on their respective datasets.}
	\label{user_vs_user}
\end{figure}

The models trained under these conditions achieved false positive rates of 1.2\% for Model 25 and 21\% for Model 86. These results represent a significant improvement over the baseline models, which demonstrated false positive rates exceeding 60\%, as shown in Figure \ref{fig:all_models_5_len40}.
This test demonstrates that the proposed method is capable of distinguishing between these two users when trained under optimal conditions. However, when the models are trained on the full dataset, this ability is substantially diminished.

\chapter{Conlusion}

In summary, the approach used in this project and study needs to be further studied to be effectively evaluated. The effects of such an approach seem promising, but are currently behind the state-of-the-art models. A larger study would also be a source of validation or rebutal to many hypotheses posed in the discussion chapter about the possible sources of error in model's performance on some specific users. Larger dataset specifically could also benefit the scientific community at large, if the users would agree to releasing it under an open source licence.

It is worth highlighting that for many users, the model performed well. Further refinement or combining the model with other neural network types, feature engineering or modification to data gathering process could also lead to improved results.

The research team considers the project successful in demonstrating that the proposed approach has merit and warrants further exploration.

%--------------------------------------
% Literatura
%--------------------------------------

\bibliographystyle{plain}{\raggedright\sloppy\small\bibliography{bibliography}}

%--------------------------------------
% Dodatki
%--------------------------------------

\cleardoublepage\appendix%

%--------------------------------------
% Informacja o prawach autorskich
%--------------------------------------

\end{document}
