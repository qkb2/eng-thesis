
\chapter{Graph Convolutional Networks - theory}

TODO for anyone/everyone (probably me and FK): how GNNs work, how GCNs work, how the networks can be constructed.

Graphs can defined as mathematical structures $$G$$ consisting of a set of vertices $$V$$, a set of edges $$E$$ and an incidence function $\phi$, along with many variations and generalizations to such structure, can be used for describing entities, which are related to each other in some way. An example of such model could be a computer network graph or citation network. Neurons can also be modelled in a similar way. Relation data can often be best described using such graphs. \cite{Lesk2024}

In modern Machine Learning, a popular type of Neural Network is a Convolutional Neural Network. Such networks generally operate on grids. A Convolutional Neural Network (CNN) has a fixed node ordering -- some input must firstly be mapped into a grid to be used with a CNN. There are ways to map many types of data into such format. For the scope of this project, \cite{Shar2023} uses such an approach to map keystroke data onto a grid, that is later used... (TODO: continue, read the paper, sth)

In this project, the goal was to use the graph networks that can naturally arise from keystroke data to -- on a graph level -- try to infere the users identity. (TODO: continue the paragraph)

TODO: how do GNNs work?

TODO: how do GCNs and GCN networks work? Network design is better left for project model section.

TODO: graph-level prediction for GCN network.