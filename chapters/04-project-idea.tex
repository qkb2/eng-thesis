
\chapter{Gathering keystroke data on mobile devices}

There are many ways to recognize a phone user using biometrics, such as scanning fingerprints or facial recognition. It is very useful for security purposes. The ease of use and reliability have made passwords less popular and led to their replacement by biometrics. However, since other biometric methods are also available, it is reasonable to test if biometrics derived from writing button press intervals and phone orientation could also be a reliable way to recognize the user. To collect data and test the results, a mobile application was created.
The main goal of the application is to gather data with an easy-to-use, intuitive interface, send the data to a server for training purposes, and check if the model recognizes the user. 

As previously stated in previous chapters, state-of-the-art models can perform well on such data \cite{Lu2020}. These models are however usually trained on data gathered from physical keyboards. Additionally, the Neural Network model created for user identification was chosen to be based on Graph Convolutional Networks, which differ from models used by many researchers in the past.
Because of that, an important part of the project was a study of results and data gathered, which is presented later.


