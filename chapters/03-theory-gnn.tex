
\chapter{Graph Convolutional Networks - theory}

Graphs can defined as mathematical structures $G$ consisting of a set of vertices $V$, a set of edges $E$ and an incidence function $\phi$, along with many variations and generalizations to such structure, can be used for describing entities, which are related to each other in some way. An example of such model could be a computer network graph or citation network. Neurons can also be modelled in a similar way. Relation data can often be best described using such graphs. \cite{Lesk2024}

Some problems relating to such data can be solved using Convolutional Neural Networks -- this can also be the case for keystroke dynamics data, such as with Lu et al. \cite{Lu2020} or Sharma et al. \cite{Shar2023}. However, it can be reasoned that the Graph Neural Networks can also perform such tasks, with connections in graph data being used more directly in the model itself.

\section{Graph Neural Networks}
TODO: how do GNN graph embeddings etc. work?

\section{Convolutional Networks and Graph Convolutional Networks}
TODO: how do GCNs and GCN networks differ? Network design is better left for project model section.

\section{Graph-level prediction in GNN}
TODO: graph-level prediction for GCN network.