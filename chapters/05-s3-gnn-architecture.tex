\section{Architecture}
The foundation of a model for graph-level prediction consists of graph convolutional layers, a pooling function to aggregate node information, and a classification head, which outputs a single value, which is then passed to a sigmoid function. 
The architecture used in this project is illustrated in figure \ref{fig:model_arch}.

\begin{figure}[H]
\begin{center}
	\begin{tikzpicture}[node distance=1.5cm]
		
		% Nodes
		\node (preff) [process] {Fully Connected + ReLu};
		\node (gcn) [process, below of=preff] {GCNConv + ReLU \\ x2};
		\node (pooling) [process, below of=gcn] {Global Mean Pooling};
		\node (linear1) [process, below of=pooling] {Fully Connected + ReLu};
		\node (dropout) [process, below of=linear1] {Dropout};
		\node (linear2) [process, below of=dropout] {Fully Connected + Sigmoid};
		
	% Background Layers
	\begin{pgfonlayer}{background}
		% Postprocessing box
		\node [draw=red!70, thick, dashed, rounded corners, fill=gray!8, inner sep=0.2cm, fit={(linear1) (dropout) (linear2)}] (postprocessing) {};
		% Preprocessing box
		\node [draw=red!70, thick, dashed, rounded corners, fill=gray!8, inner sep=0.2cm, fit={(preff)}] (preprocessing) {};
	\end{pgfonlayer}
	
	% Labels for groups
	\node [right=0.2cm of postprocessing, text=red] {Postprocessing layers};
	\node [right=0.2cm of preprocessing, text=red] {Preprocessing layers};

		
		% Arrows
		\draw [arrow] (preff) -- (gcn);
		\draw [arrow] (gcn) -- (pooling);
		\draw [arrow] (pooling) -- (linear1);
		\draw [arrow] (linear1) -- (dropout);
		\draw [arrow] (dropout) -- (linear2);
		
	\end{tikzpicture}
	
	\caption{Model architecture}
	\label{fig:model_arch}
\end{center}
\end{figure}

\subsection{GCN + ReLU}
The first two layers of the model, pictured in figure \ref{fig:model_arch}, are graph convolutional layers or GCNConv layers. The theory behind these layers was extensively discussed in chapter 3. After each layer a ReLU activation function was applied.

\subsection{Global Mean Pooling}
After two convolutional layers, a global mean pooling layer was applied to aggregate the node embeddings into a single vector. 


\subsection{Preprocessing layers}
Considering the complexity of the input features, specifically the relationship between parts of the features used to represent characters discussed in subsection \ref{Base_letter_representation}, a preprocessing fully connected layer was added to the model. Preprocessing layers can improve performance in cases where encoding node features is necessary, such as text inputs \cite{Lesk2024}.

\subsection{Postprocessing layers}
The selected architecture uses two fully connected layers to transform the aggregated output into a single value, 
which is passed through a sigmoid activation function. The first fully connected layer was added after empirical experimentation which showed an improvement in performance.


\subsection{Dropout}
A dropout layer was added between the two fully connected layers, which randomly disable some of the neurons during the training process. Such layers have been shown to create more robust models, that generalize better, as demonstrated by \cite{NIPS2013_71f6278d}. Additionally, dropout layers have also been found to improve expressiveness in GNNs \cite{NEURIPS2021_b8b2926b}.

\section{Training and fine-tuning}

\subsection{Data division}
For model validation, 5-fold cross-validation was used, splitting the training set described in section \ref{sec:data_collection} into five parts, with training performed on four of them. The number of folds corresponds to the number of 300-character blocks that users were asked to input. Such splits validate that the trained model is able to generalize to a part of the input that may have been written at different times and in styles. Furthermore, it closely resembles the way the model was tested, that is on a completely different input sequence, collected in a separate part of the application.  

\subsection{Training parameters}

\subsection{Loss function}
Binary Cross Entropy was chosen as the loss function for training, as it is commonly used with binary classifiers. 
The use of this loss function requires the model to output the probability of picking the positive class, therefore a sigmoid function must be applied to the output of the final layer.
These two operations are combined into a single step in the implementation, by using  $BCEWithLogitsLoss$, which provides greater numerical stability \cite{Ansel_PyTorch_2_Faster_2024}.

\subsection{Optimizer}
Adam was selected as the optimizer for updating model weights, as it adapts the learning rate for each parameter. The learning rate was empirically set to 0.001.

\subsection{Tuning hyperparameters}
The architecture described above allows for a large number of hyperparameter choices, such as the size of preprocessing, convolutional, and postprocessing layers. Furthermore, the variable length of input sequences and multiple possible feature encoding methods mean that a comprehensive search over the hyperparameter space is infeasible. Within the scope of this project the biggest emphasis was placed on finding the optimal input features and input sequence length, thereby identifying the best graph representation. Additionally, the effect of class imbalance on the training process of the collection of models was also explored.\\
All models were trained using 5--fold cross--validation, with the validation results averaged across folds. These were then used to compare the collections of models and to select hyperparameter values.