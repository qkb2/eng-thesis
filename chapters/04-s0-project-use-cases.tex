
\section{Use cases}
The main goal of this application was the identification of users based on their distinctive typing behavior, which is known as keystroke dynamics. By using machine learning models and encrypted server transmission for analyzing the collected data, the application aimed to provide an additional layer of security beyond passwords or basic biometrics. This project tried to establish whether this type of behavioral biometric can be a reliable way of user authentication. \newline
The following use cases illustrate how the user would interact with the application and its features.

\begin{itemize}
	\item \textbf{Logging into the application}
	\begin{itemize}
		\item \textbf{Purpose:} Allowing the user to log in and associate the application with a predefined \texttt{ID}.
		\item \textbf{Steps:}
		\begin{itemize}
			\item The user opens the application.
			\item On the \texttt{Login Screen}, the user enters their \texttt{ID} in the input field.
			\item The user clicks the \texttt{Login} button to proceed.
			\item The application stores given \texttt{ID} for further operations.
		\end{itemize}
		\item \textbf{Result:} The user is logged in and is redirected to the \texttt{Home Screen}.
	\end{itemize}
	
	\item \textbf{Data collection from key presses}
	\begin{itemize}
		\item \textbf{Purpose:} Storing users' keyboards interaction data for analysis.
		\item \textbf{Steps:}
		\begin{itemize}
			\item The user navigates to \texttt{Training Screen}.
			\item The user types a predefined number of characters in total throughout 5 phases to complete training.
			\item The application registers data for every key press, including:
			\begin{itemize}
				\item Key ID (e.g., \texttt{A}, \texttt{h}, \texttt{3})
				\item Timestamp of key press action
				\item Press duration
				\item Accelerator data (X, Y, Z axis)
			\end{itemize}
			\item Data is stored in \texttt{KeyPressEntity} object for further processing.
		\end{itemize}
		\item \textbf{Result:} Full key press data is saved in the application, ready to be transformed into \texttt{TSV} format, transmitted to the server, or stored locally.
	\end{itemize}
	
	\item \textbf{Testing how well the model recognizes the user}
	\begin{itemize}
		\item \textbf{Purpose:} Verifying if the user entering data is the one associated with their \texttt{ID}.
		\item \textbf{Steps:}
		\begin{itemize}
			\item The user navigates to \texttt{Testing Screen}.
			\item The user types a predefined number of characters into the input field.
			\item The application registers the key press data.
			\item The data is transformed into TSV format, stored locally, and sent to the server:
			\begin{itemize}
				\item The application ensures the connection is secured with \texttt{SSL/TLS}.
				\item The \texttt{POST} method is used to send the data.
			\end{itemize}
			\item The server processes the data using the trained model.
			\item The server sends back a response to the application, including:
			\begin{itemize}
				\item Information on whether the user was recognized.
				\item The percentage of compliance with the user.
			\end{itemize}
		\end{itemize}
		\item \textbf{Result:} The application displays the result to the user.
	\end{itemize}
	
\end{itemize}