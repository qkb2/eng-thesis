
\chapter{Implementing keystroke data on mobile devices}

There are many ways to recognise phone user using biometrics, such as scanning fingerprints or facial recognition. It's very useful for security purposes. The ease of use and reliability have made passwords less popular and led to their replacement by biometrics. However, since other biometrics are also available, it is reasonable to test if biometrics derived from writing button press intervals and phone orientation could also be a reliable way to recognise the user. To collect data and test the results, the mobile application was created.
The main goal of the application is to gather data with an easy-to-use, intuitive interface, send the data to a server for training purposes, check if the model recognises the user. 

As previously stated, State of the Art models can actually perform well* on such data. These models are however usually trained on data gathered from physical keyboards. Additionally, the Neural Network model created for user identification was chosen to be based on Graph Convolutional Networks, which are still rather novel*.
Thanks to that, an important part of the project was a study of results and data gathered, which is presented in chapter 3.4 and 3.5.

TODO: find statistics and add them to sources