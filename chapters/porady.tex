Przegląd literatury naświetlający stan wiedzy na dany temat obejmuje rozdziały pisane na podstawie
literatury, której wykaz zamieszczany jest w części pracy pt.~\emph{Literatura} (lub inaczej \emph{Bibliografia},
\emph{Piśmiennictwo}). W tekście pracy muszą wystąpić odwołania do wszystkich pozycji zamieszczonych w
wykazie literatury. \textbf{Nie należy odnośników do literatury umieszczać w stopce strony.} Student jest
bezwzględnie zobowiązany do wskazywania źródeł pochodzenia informacji przedstawianych w pracy,
dotyczy to również rysunków, tabel, fragmentów kodu źródłowego programów itd. Należy także podać
adresy stron internetowych w przypadku źródeł pochodzących z Internetu.

Rozdziały dokumentujące pracę własną studenta: opisujące ideę, sposób lub metodę 
rozwiązania postawionego problemu oraz rozdziały opisujące techniczną stronę rozwiązania 
--- dokumentacja techniczna, przeprowadzone testy, badania i uzyskane wyniki. 

Praca musi zawierać elementy pracy własnej autora adekwatne do jego wiedzy praktycznej uzyskanej w
okresie studiów. Za pracę własną autora można uznać np.: stworzenie aplikacji informatycznej lub jej
fragmentu, zaproponowanie algorytmu rozwiązania problemu szczegółowego, przedstawienie projektu 
np.~systemu informatycznego lub sieci komputerowej, analizę i ocenę nowych technologii lub rozwiązań
informatycznych wykorzystywanych w przedsiębiorstwach, itp. 

Autor powinien zadbać o właściwą dokumentację pracy własnej obejmującą specyfikację założeń i 
sposób realizacji poszczególnych zadań
wraz z ich oceną i opisem napotkanych problemów. W przypadku prac o charakterze 
projektowo-implementacyjnym, ta część pracy jest zastępowana dokumentacją techniczną i użytkową systemu. 

W pracy \textbf{nie należy zamieszczać całego kodu źródłowego} opracowanych programów. Kod źródłowy napisanych
programów, wszelkie oprogramowanie wytworzone i wykorzystane w pracy, wyniki przeprowadzonych
eksperymentów powinny być umieszczone np. na płycie CD, stanowiącej dodatek do pracy.

\section*{Styl tekstu}

Należy stosować formę bezosobową, tj.~\emph{w pracy rozważono ......, 
w ramach pracy zaprojektowano ....}, a nie: \emph{w pracy rozważyłem, w ramach pracy zaprojektowałem}. 
Odwołania do wcześniejszych fragmentów tekstu powinny mieć następującą postać: ,,Jak wspomniano wcześniej, ....'', 
,,Jak wykazano powyżej ....''. Należy unikać długich zdań. 

Niedopuszczalne są zwroty używane w języku potocznym. W pracy należy używać terminologii informatycznej, która ma 
sprecyzowaną treść i znaczenie. 

Niedopuszczalne jest pisanie pracy metodą \emph{cut\&paste}, bo jest to plagiat i dowód intelektualnej indolencji autora.
Dane zagadnienie należy opisać własnymi słowami. Zawsze trzeba powołać się na zewnętrzne źródła. 

Zakończenie pracy zwane również Uwagami końcowymi lub Podsumowaniem powinno zawierać ustosunkowanie
się autora do zadań wskazanych we wstępie do pracy, a w szczególności do celu i zakresu pracy oraz
porównanie ich z faktycznymi wynikami pracy. Podejście takie umożliwia jasne określenie stopnia
realizacji założonych celów oraz zwrócenie uwagi na wyniki osiągnięte przez autora w ramach jego
samodzielnej pracy.

Integralną częścią pracy są również dodatki, aneksy i załączniki zawierające stworzone w ramach pracy programy, aplikacje i projekty.