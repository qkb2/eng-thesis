\chapter{GCN Model}

In this project, the goal was to use the graph networks that can naturally arise from keystroke data to -- on a graph level -- try to infere the users identity. A key characteristic of a project was that the use of any keyboard already installed on the user's mobile phone causes some problems with gathering keystroke temporal data, as mentioned in the second chapter. Because of that, this project used data involving only Up-Up times, with additional use of accelerometer data being considered by the researchers and discussed in the next section. \myworries{what data was actually mapped?}

Moreover, this project focused on creating an collection of models, one model for each user that performs binary classification rather than one large model for multiclass classification. 
This decision was made for several reasons. Firstly, such scheme allows for models to be trained on demand, as soon as a new user provides all the training data to the mobile application. Secondly, new users do not force the whole model to be retrained, 
as only one new model needs to be created, and provided all models have learned their target users sufficiently, they would be able to reject such new users without further tuning. Lastly, the one user per model scheme allows for inference to take place locally, on the target user's device. This would remove the need for remote communication with the server, thus increasing the mobile application's reliability and security. On device inference is an area of active research \cite{48305}. However, the complexity of such solution was deemed to great and outside the scope of this project.
