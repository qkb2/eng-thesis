\chapter{GCN Model}

In this project, the goal was to use the graph networks that can naturally arise from keystroke data to infer the user's identity at a graph level. A key characteristic of a project was that the use of any keyboard already installed on the user's mobile phone causes some issues with gathering keystroke temporal data, as mentioned in the second chapter. Because of that, this project used data involving only Up-Up times, with additional use of accelerometer data being considered by the researchers and discussed in the next section.

Moreover, this project focused on creating a collection of models, one model for each user that performs binary classification rather than a single large model for multiclass classification. This decision was made for several reasons. First, this scheme allows for models to be trained on demand, as soon as a new user provides all the training data to the mobile application. Second, new users do not require retraining of the entire model, as only one new model needs to be created, and provided all models have learned their target users sufficiently, they would be able to reject such new users without further tuning. Finally, the one user per model scheme allows for inference to take place locally, on the target user's device. This would eliminate the need for remote communication with the server, thus increasing the mobile application's reliability and security. On-device inference is an area of active research \cite{48305}. However, the complexity of such a solution was deemed too great and beyond the scope of this project.
